\documentclass[open=left]{Ilaris}
\usepackage{pdfpages}
\usepackage{verbatim}
\usepackage{fourier}
\usepackage{geometry}


\newcommand{\iconsvg}[1]{{\raisebox{-0.2\baselineskip}{\includesvg[height=\baselineskip]{#1}\hspace{0.1cm}}}}%0.1
%\newcommand{\icon}[1]{\includesvg[height=\baselineskip]{#1}\hspace{0.1cm}}
\newcommand{\correct}{\iconsvg{res/img/correct}}
\newcommand{\incorrect}{\iconsvg{res/img/incorrect}}
\newcommand{\difficult}{\iconsvg{res/img/Schwierigkeit_schwierig}}
\newcommand{\easy}{\iconsvg{res/img/Schwierigkeit_einfach}}
\newcommand{\additional}{\iconsvg{res/img/Text_Zusatz}}
\usepackage{svg}
\usepackage{listings}
\lstset
{
	language=[LaTeX]TeX,
	breaklines=true,
	basicstyle=\texttt\scriptsize,
	keywordstyle=\color{dunkelrot},
	identifierstyle=\color{braun},
}

\setdescription{itemsep=0cm} %\baselineskip
\setenumerate{itemsep=0cm} %\baselineskip


\graphicspath{%
	%Bilder liegen in Unterordnern. Hier Windows-Codiert. Für bessere Betriebssysteme \ mit / ersetzen.
	%Je nach Ordnerstruktur müssen auch im ersten Pfad die ..\angepasst werden.
	{/home/mot/Dropbox/Rollenspiele/DSA/Ilaris/Wettbewerb/Anthologie/res/img/}%
}

\title{Festtagsschmaus}

\begin{document}
	\hauptteil
%	\begin{center}
%		\zeichnung{header_tairach.png}
%		%\color{dunkelrot}\fontsize{16}{16}\aniron%
%		%	von Tilmann Kircher\\[0mm]
%		%\zeichnung{Tairach_Icon.png}
%		%	\normalcolor
%		%	\normalfont
%		
%		\usekomafont{subsection}
%		Ein Abenteuer für 3 - 6 angehende Heldinnen und Helden (etwa 2.500 EP)
%		\normalsize
%		\normalfont
%		\normalcolor
%		
%		\platz
%	
%\begingroup
%\let\clearpage\relax
%\tableofcontents
%\endgroup
%
%
%\platz
%
%	\abschnitt{Vorwort}
%	
%	\geschichte{%
%		Vor mehr als 30 Jahren wurden in der ersten Spielhilfe zu Thorwal die orkischen Hafenarbeiter der Stadt Thorwal eingeführt.
%		Seitdem sind sie immer wieder mal erwähnt worden, wir haben ihre genauen Lebensumstände jedoch nie erfahren können.
%		Mich hat jetzt schon eine Weile die Frage umgetrieben, wie die eigentlich patriarchalischen Orks
%		(das Wort für \anf{Frau} auf Ologhaijan bedeutet übersetzt \anf{Tier das Orks gebiert})
%		innerhalb der Thorwalschen Gesellschaft mit Orkinnen umgehen. Laufen Orkinnen wie im Orkland unbekleidet herum oder kleiden sie sich wie menschliche Frauen Thorwals, bleiben sie ohne Namen oder haben sie zumindest einen an ihren Vater angelehnten Namen? Was geschieht, wenn ein Schamane, der Kontakt zur thorwalschen Kultur hat, magische Fähigkeiten bei einem Orkmädchen bemerkt? Sollte er darin nicht einen Wink Tairachs sehen? Nachdem ich diese Idee weitergesponnen habe, ist dieses Abenteuer dabei entstanden.
%	}{Tilmann Kircher, im Januar 2022}
%	
%	\platz
%
%	Mit Dank an meine Testspielerinnen und Testspieler\\
%	Daniel Gunzelmann, Stefan Ivenz, Jan Klose, Peter Lichtenwagner, Melanie Lippel,\\  Ralf Luger, Marle Müller, Julian Römer und Jörg Rüdenauer
%	
%\end{center}
%
%\platz
%	
%	
%	
%	\subsection*{Version 2.0 vom \today}
%
%Das Abenteuer wurde erstmals im Sommer 2022 in \enquote{Die Chroniken von Ilaris -- Band 1} veröffentlicht.
%Für die Ilaris-Box wurde es alleinstehend gesetzt. Seitenzahlen weichen vom Original ab, Text und übrige Inhalte sind unverändert.
%		
%	\platz
%	\credit{Autor}{Tilmann Kircher%\\
%%		\href{https://github.com/XaverStiensmeier/IlarisGlossaryEntries}{Xaver Stiensmeier}\\
%	%	Rothar
%	}
%
%	\credit{Illustrationen}{\href{https://www.artstation.com/bernhard\_eisner}{Bernhard Eisner}\\
%		%	Fenia Winterkalt\\
%	%	Norman Obst\\
%	%	René Dudziak\\
%%		\href{https://www.zillie-zeh.de/}{Zillie Zeh}
%	}
%	
%%	\credit{Fotografien}{Lukas Ruhe}	
%	
%	\credit{\LaTeX-Klasse}{Lukas Ruhe}
%	
%%	\credit{Edition}{Matthias Ott
%%	%	\\	\href{https://github.com/XaverStiensmeier/IlarisGlossaryEntries}{Xaver Stiensmeier}
%%	}
%	
%	\credit{Layout}{Daniel Gunzelmann\\Matthias Ott}
%	
%	\credit{Lektorat und Korrektorat}{Stefan Ivenz\\Jörg Rüdenauer\\Matthias Ott}
%	
%%	\credit{Korrektorat}{%\\Tilmann Kircher
%%	}
%	%\credit{Glossar}{\href{https://github.com/XaverStiensmeier/IlarisGlossaryEntries}{Xaver Stiensmeier}}
%	\lizenz
%	
%	%\fanprodukt
%\neueseite
%	
%	\spaltenanfang
%	
%	\input{res/tex/tairach}
%	
%	\spaltenende


		\subsection*{Version 2.0 vom \today}

Das Abenteuer wurde erstmals im Sommer 2022 in \enquote{Die Chroniken von Ilaris -- Band 1} veröffentlicht.
Für die Ilaris-Box wurde es alleinstehend gesetzt. Seitenzahlen weichen vom Original ab, Text und übrige Inhalte sind unverändert.

\platz
\credit{Autor}{\href{https://github.com/XaverStiensmeier/IlarisGlossaryEntries}{Xaver Stiensmeier}\\
	%	Rothar
}

\credit{Illustrationen}{\href{https://www.artstation.com/bernhard\_eisner}{Bernhard Eisner}\\
	%	Fenia Winterkalt\\
	Norman Obst\\
	René Dudziak\\
	%		\href{https://www.zillie-zeh.de/}{Zillie Zeh}
}

%	\credit{Fotografien}{Lukas Ruhe}	

\credit{\LaTeX-Klasse}{Lukas Ruhe}

\credit{Edition}{Matthias Ott
	\\	\href{https://github.com/XaverStiensmeier/IlarisGlossaryEntries}{Xaver Stiensmeier}
}

%	\credit{Layout}{Daniel Gunzelmann\\Matthias Ott}

\credit{Korrektorat}{Tilmann Kircher\\Matthias Ott}

%	\credit{Korrektorat}{%\\Tilmann Kircher
%	}
%\credit{Glossar}{\href{https://github.com/XaverStiensmeier/IlarisGlossaryEntries}{Xaver Stiensmeier}}
\lizenz

%\fanprodukt
\neueseite

	\begin{center}
		\zeichnung{header_schmaus.PNG}
		
	
		\slshape
		\geschichte{Durch schlechte Köche -- durch den vollkommenen Mangel an Vernunft in der Küche ist die Entwicklung des Menschen am längsten aufgehalten, am schlimmsten beeinträchtigt worden: es steht heute selbst noch wenig besser.\newline
			Ein Bissen guter Nahrung entscheidet oft, ob wir mit hohlem Auge oder hoffnungsreich in die Zukunft schauen.}{aus den Chroniken von Ilaris}
		
		\platz
	
\begingroup
\let\clearpage\relax
\tableofcontents
\endgroup
\platz
	\end{center}
	\normalfont
	
	\section*{Danksagung}
	
	Danke an Ilaris, Lukas (Curthan) Schafzahl, Gurrgak, Das Schwarze Auge und natürlich meiner Spielerschaft. Besonderer Dank gebührt der Weihnachtsschließung der Universität, ohne die ich keine Zeit für den Wettbewerb gefunden hätte.
	\platz
	%	\noindent\fbox{%
		%	\parbox{0.5\linewidth}{%~0.5 wenn das blöd aussieht, abändern ;)
			\begin{center}
				
				\bild[8cm]{meister-handouts.png}
				
			\end{center}
			%		}%
		%	}
	
	\neueseite

	\spaltenanfang
	\input{res/tex/schmaus.tex}
	

	
	
%	\addcontentsline{toc}{chapter}{3.\,Platz: \anf{Ein Abend, \newline teuer für Einsteiger}}
%	
%	\begin{center}
%		\zeichnung{header_abend.PNG}
%		
%		\usekomafont{section}
%		\bigskip
%		Jänner 2022
%	\end{center}
%	
%	
%	
%	\platz
%	
%	\section*{Danksagung}
%	
%	\normalcolor
%	\normalfont
%	\normalsize
%	
%	Mein Dank gebührt Chris, Christian, Daniel, Gregor, Kathi, Kurt, Martina, Nadine, Renate, Wolf und Wolfgang mit denen ich nun schon seit drei Jahrzehnten in Aventurien, auf Entaria oder so mancher anderer phantastischer Welt um die Häuser ziehe. Und meiner Frau Maria, die viel Verständnis für diese Leidenschaft zeigt.
%	
%	
%	\platz
%	
%	\neueseite
%	
%	\spaltenanfang
%	\input{res/tex/abend}
%	\spaltenende

\neueseite

\part*{Notizen}
	
\end{document}
