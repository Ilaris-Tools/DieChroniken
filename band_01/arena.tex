\abschnitt{Einleitung}
Dieses Szenario wurde geschrieben damit die Ilaris-Regeln für den Kampf und die Heilung ausgiebig ausprobiert werden können. Um hier gewisse Freiheiten zu gewinnen, wurde ein „Abenteuer im Schlaf“ gewählt.

„Im Traum“ übernimmt jeder Spieler für eine Szene einen der vorgefertigten Charaktere. Für jede Szene kann neu gewählt werden. 

Die Traumgestalten werden mit jeder Szene mächtiger. Das heißt Geschicklichkeit, Körperkraft, Konstitution usw. werden nach Erfordernis erhöht und höherwertige Manöver werden den Fähigkeiten hinzugefügt. 

Vor jedem Kampf gibt es eine Besprechung mit einem Trainer, der die Taktik inklusive Manöver und Ausrüstung bespricht.

Diese (reduzierten) Traumgestalten stehen für dieses Abenteuer zur Auswahl. Sie werden nicht vollwertig als Archetypen ausgearbeitet.

\kreatur{Tank}{KK, KO}{schild.png}{ 
\kreaturinfo{Kampfstil (KK)}{Schildkampf}
\kreaturinfo{Manöver}{Schildwall}
\kreaturinfo{Kampfvorteile (KO)}{Zäher Hund, Rüstungsgewöhnung, Eisern, Verbesserte Rüstungsgewöhnung}
\kreaturinfo{Kampfvorteile (KK)}{Manöver: Niederwerfen}
}

\kreatur{Support}{CH, IN}{rondra.png}{ 
\kreaturinfo{Aktion (Autorität)}{Einschüchtern}
\kreaturinfo{Aktion (Heilkunde)}{Heilung nach dem Kampf}
\kreaturinfo{Kampfvorteile (CH)}{Kommandos }
\kreaturinfo{Kampfvorteile (IN)}{Kampfreflexe, Defensiver Kampfstil, Aufmerksamkeit, Klingentanz}
\kreaturinfo{Manöver}{Klingentanz}
\kreaturinfo{Klasse}{Barden, Zauberer, Geweihte}
}
\kreatur{Grobian}{KK}{kraftvoll.png}{
\kreaturinfo{Kampfstil (KK)}{Kraftvoller Kampf }
\kreaturinfo{Manöver}{Befreiungsschlag}
\kreaturinfo{Kampfvorteil (KK)}{Niederwerfen, Waffenloser Kampf, Hammerschlag, Unaufhaltsam}
\kreaturinfo{Manöver}{Niederwerfen, Hammerschlag }
}
\kreatur{Blitz}{GE, IN}{schnell.png}{
\kreaturinfo{Kampfstil}{Schneller Kampf}
\kreaturinfo{Manöver}{Unterlaufen }
\kreaturinfo{Kampfvorteile (GE)}{Standfest, Sturmangriff, Todesstoß}
\kreaturinfo{Manöver}{Sturmangriff, Todesstoß}
\kreaturinfo{Kampfvorteile (IN)}{Kampfreflexe, Defensiver Kampfstil, Aufmerksamkeit}
}
\kreatur{Fechter}{GE}{Parierwaffen.png}{
\kreaturinfo{Kampfstil}{Parierwaffenkampf}
\kreaturinfo{Manöver}{Riposte}
\kreaturinfo{Kampfvorteile (GE)}{Standfest, Sturmangriff, Todesstoß}
\kreaturinfo{Manöver}{Sturmangriff, Todesstoß}
}
\kreatur{Doppel}{GE, MU}{beidhändig.png}{ 
\kreaturinfo{Kampfstil}{Beidhändiger Kampf}
\kreaturinfo{Manöver}{Doppelangriff }
\kreaturinfo{Kampfvorteile (GE)}{Standfest, Sturmangriff, Todesstoß, Präzision}
\kreaturinfo{Manöver}{Sturmangriff, Todesstoß}
\kreaturinfo{Kampfvorteile (MU)}{Offensiver Kampfstil}
\kreaturinfo{Manöver}{Ausfall, Gegenhalten}
}
\kreatur{Schütze}{FF}{wurfwaffen.png}{ 
\kreaturinfo{Kampfvorteile (FF)}{Ruhige Hand, Reflexschuss, Schnellziehen, Meisterschuss }
\kreaturinfo{Manöver}{Reflexschuss, Schnellschuss, Meisterschuss, Rüstungsbrecher}
}

\begin{center}
	\bild[5.5cm]{Kor.png}
\end{center}

Die Charaktere liegen sanft schlafend in ihren Betten. Niemand geringerer als \begriff{Kor}, der Gott des Kampfes entführt sie in dieser Nacht in ihren Träumen, um zu prüfen, ob sie in den Kämpfen dieser Welt bestehen können.


\abschnitt{Szene 1: Die Wahl der Waffen}
Erklärung der Generierungsrichtlinien und Absichten

Wahl der ersten Traumgestalt, Reihenfolge der Auswahl für die Szenen

Wahl von Rüstung und Waffen

Anwendung von Schicksalspunkten im Kampf

\spaltenende
\zeichnung{waffen.jpg}
\neueseite
\spaltenanfang

\abschnitt{Szene 2: Eine Hundemeute}
Grundsätzliches zu Aktionen, Ini, AT, VT, Mehrere VT’s, Bewegung, Umgebung nutzen, Fernkampf, Triumph und Patzer, Kommandos, Haltungen (Volle Offensive bzw. Defensive)

\neueseite 

\abschnitt{Szene 3: Heilung}
Anwendung von profaner, magischer und karmaler Heilung

\zeichnung{humus.jpg}

\zeichnung{heilung.png}

\abschnitt{Szene 4: Steingolem}
Angriff mit Ansage gegen schwer Gepanzerte, Rüstungsbrecher, Niederschmettern

\spaltenende

\zeichnung{massenkampf.jpg}
\neueseite


\spaltenanfang
\abschnitt{Szene 5: Schützenfest}
Kampf gegen Bogenschützen, Modifikationen, Deckung

\zeichnung{schuetzin.JPG}
\neueseite

\abschnitt{Szene 6: Ein Überfall in Unterzahl}
Überraschung, „richtiges“ Meucheln

\zeichnung{ueberfall.jpg}