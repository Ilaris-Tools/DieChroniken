\platz

\credit{Autor}{Matthias Ott}

\credit{Illustrationen}{\href{https://www.artstation.com/bernhard\_eisner}{Bernhard Eisner}\\game-icons.net
}
\credit{\LaTeX-Klasse}{Lukas Ruhe}

\credit{Mit Dank an}{Kilian Linder für konstruktives Feedback,\\ Xaver Stiensmeier für die Ausrichtung des Wettbewerbs und Odir Sensendengler\\ sowie die geduldige Testgruppe: Jascha, Gloria, Meike und Hannes
}

\newpage
\spaltenanfang


\addcontentsline{toc}{section}{Einleitung\,für\,die\allowbreak\ Spielleitung}
\section*{Einleitung für die Spielleitung}
In diesem Abenteuer kann eine Gruppe in \textbf{Finsterkoppen},
der Hauptstadt des Bergkönigreichs eines zwergischen Volkes im \textbf{Finsterkamm},
eine Intrige um die Einsetzung eines Hochkönigs aufdecken und damit die Spaltung in zwei Völker auslösen.

Dies ist auch der Plot von \emph{Im Traumlabyrinth} (Schmidt Spiele, 1990).
Während die Ergebnisse dieses Abenteuers in jüngeren Publikationen (wie \emph{Angroschs Kinder} (Fanpro, 2005))
übernommen wurde, passen die Inhalte des Abenteuers selbst nicht in das später entwickelte Aven\-turien\-bild und ergeben auch keinen spannenden Spielabend.
Ich wollte daher die Möglichkeit bieten, die historische Entwicklung im Rahmen einer anderen Handlung miterleben zu können
und dabei Orte und Figuren kennen zu lernen, die in später stattfindenden Abenteuern wieder eine Rolle spielen (können).

Den Schauplatz \textbf{Finsterkoppen} und die \textbf{Finsterkopp-Binge} waren schon im Computerspiel \emph{Sternenschweif} (Attic, 1994) zu erkunden. %, das ich vor etwa 20 Jahren das letzte Mal gespielt habe.
%Was mir davon noch in Erinnerung war, habe ich verarbeitet.
Ich habe für den Anhang auf die Liste der Gebäude aus \link{https://www.kunar.eu/nlt/finsterkoppen.htm}{Kunars Reiseführer} zurückgegriffen.
Die Idee von Rattengold als überschwerem 13.\,Unmetall haben Simon Schollenberger, Stefan Weber und Stefan Löffler für das Wettbewerbs-Abenteuer \enquote{\link{https://www.orkenspalter.de/filebase/index.php?file/53-totgeboren/}{Totgeboren}} entwickelt.
Auch wenn sie den Weg ins kanonische Aventurien noch nicht gefunden hat, habe ich sie hier aufgegriffen.


\subsection*{Aufbau und Verlauf}
Ich habe das Abenteuer in drei Akte unterteilt, die nach meiner Einschätzung jeweils eine bis zwei Stunden Spielzeit ergeben sollten.
Sie bauen inhaltlich aufeinander auf.
Neben den beschriebenen Schlüsselszenen kann Ihre Gruppe durch eigene Schwerpunktsetzungen die Handlung mitbestimmen -- beispielsweise,
indem sie im \emph{ersten Akt} mit der Bevölkerung von \textbf{Finsterkoppen} Handel treibt oder im \emph{zweiten Akt} weitere Räumlichkeiten erkundet.
Wegen des beschränkten Platzes müssten Sie die dazugehörigen Szenen jedoch gemeinsam improvisieren.

Zu Ihrer Orientierung ist zu Beginn jedes Aktes angegeben, welche Ziele die Gruppe hier verfolgen (und gegebenenfalls erreichen) sollte.
Ebenso gebe ich hier Tipps, wie Sie -- falls die Zeitplanung für Sie nicht passt  oder der Abschnitt Ihrer Gruppe keinen Spaß machen würde -- ein längeres oder kürzeres Spiel erreichen könnten.

Im \emph{ersten Akt} wird die Gruppe an das Problem herangeführt:
Die Gruppe bringt die Nachricht einer äußeren Gefahr in die Stadt.
Um dieser zu begegnen, würde der \textbf{Tiefe Rat} der Finsterkamm-Zwerge gern einen Hochkönig wählen.
Der greise Bergkönig \textbf{Gerambolosch}, der das Verfahren einleiten muss (und der natürliche Kandidat für dieses Amt wäre), ist jedoch nicht ansprechbar -- wie es scheint, aus Altersgründen.
Die einzige Alternative ist, das \textbf{Szepter der Stärke} durch den örtlichen Hochgeweihten des Angrosch neu schmieden zu lassen.
Das Szepter lagert jedoch in einer alten Binge, die die Finsterkamm-Zwerge nicht mehr betreten wollen.

\kasten{\enquote{Binge} ist ein Fantasie-Wort, das zwergische unterirdische Anlagen  beschreibt, die sowohl Bergwerk als auch Wohnsiedlung umfassen.}

Die Suche führt die Gruppe im \emph{zweiten Akt} in ebendiese Hallen.
Hier gilt es, Fallen zu entgehen, Wühlschrate zu ertragen und schließlich ein Rätsel zu entschlüsseln, um Zugang zum Szepter zu erhalten.
Eine Begegnung mit einem \textbf{Bosnickel} gibt den Hinweis, dass das Unmetall \textbf{Rattengold} in der Binge zu finden ist.
Die Gruppe entdeckt, dass jemand davon an die Oberfläche bringt.
\kasten{\enquote{Unmetalle} sind Materialien, die in der Regel Dämonen zugeordnet werden. Einzige Ausnahme ist das hier vorkommende, dem Namenlosen zugeordnete \textbf{Rattengold}. Ihre Eigenschaften sind nicht festgelegt.}

Im \emph{dritten Akt} versucht die Gruppe herauszufinden, wer dies gewesen sein könnte.
So kann es ihnen gelingen, während der Hochkönigwahl den Intriganten \textbf{Bonderik} zu enttarnen,
der diese zu seinen Gunsten beeinflussen wollte.

\subsection*{Was wird zum Spielen benötigt?}
Für erfahrene Spielleitungen dürfte das Abenteuer selbst sowie die Ilaris-Regeln als Grundlage zur Gestaltung des Spielabends ausreichen.
Wenn Sie mit der Welt Aventurien noch nicht viel Kontakt hatten, empfehle ich zur Vorbereitung den Band \emph{Angroschs Kinder} (Fanpro, 2005, im folgenden \emph{AK}).
Dieser Band ist leider vergriffen und gebraucht kaum erhältlich.
Günstiger ist \emph{Die Zwerge Aventuriens} (Schmidt Spiele, 1993, im folgenden \emph{DZA}) gebraucht zu bekommen,
das jedoch älter und weniger umfangreich ist.

Ebenso empfehle ich weniger erfahrenen Spielleitungen einen Blick in eine Spielhilfe zum Thema Dungeon.
\emph{Katakomben und Kavernen} (Ulisses, 2010, im folgenden \emph{KK})
ist noch als ebook erhältlich.
Die kostenfreie Spielhilfe \enquote{\link{https://www.orkenspalter.de/filebase/index.php?file/2724-fackelschein-und-kellerstaub/}{Fackelschein und Kellerstaub}} von Ulrich Lang
zum selben Thema ist meines Erachtens ähnlich nützlich.

Ich habe darauf verzichtet, eine Karte der Finsterkopp-Binge anzufertigen, weil die genaue Platzierung der beschriebenen Räume zueinander keine Rolle spielt und ich mir die Freiheit erhalten wollte, sie nach dramaturgischen Gesichtspunkten spontan umzuverteilen. Wenn Ihre Gruppe großen Spaß an Karten hat oder Sie mehr Anregungen für die Ausgestaltung der Details suchen,
finden Sie in \textbf{Malatosch} (\emph{KK:147}) ein geeignetes Vorbild für zwergische Architektur.

\subsection*{In eine andere Zeit verlegen?}
Unsere Testspiel-Gruppe hat die Dringlichkeit einer Hochkönigwahl selbst hergestellt, indem sie die Botschaft von einer drohenden Orkgefahr nach Finsterkoppen brachte.
Mit dieser Prämisse lässt sich das Abenteuer problemlos im Zeitraum vom Auftauchen Ashims 1003 bis zum Einfall der Orken ins Svellttal 1010\,BF verschieben
und etwa mit der Suche nach dem  \textbf{Salamanderstein} (vgl. \emph{Sternenschweif}) verbinden.

Wenn Sie näher an der aventurischen Jetztzeit spielen, sich jedoch eng am aventurischen Kanon orientieren, sitzt der Hochkönig des \fkvs fest im Sattel,
während sein Konkurrent sich als Anführer der \textbf{Finsterzwerge} in der südlichen Hälfte des \textbf{Finsterkamm}-Gebirges eingerichtet hat.
Nach dem dritten Orkensturm sollten Sie die Ereignisse also in einer anderen Zwergenstadt spielen lassen und die Namen der Nichtspielerfiguren austauschen.

Die äußere Gefahr, die Anlass der Hochkönigwahl wird -- bei mir ein Orkangriff --, tritt im Abenteuer noch nicht ein. Sie können sie für Ihre Gruppe frei festlegen.
Ein Drache, eine Naturkatastrophe oder eine in der Nachbarschaft grassierende Seuche wären beispielsweise andere Möglichkeiten, die die Zwerge bewegen können, einen Hochkönig zu wählen.

\subsection*{Welche Gruppen sind geeignet?}
Das Abenteuer ist für eine Gruppe von rund 4.000 Erfahrungspunkte nach dem Ilaris-System ausgelegt. Vorgefertigte Beispiel-Figuren finden Sie im Anhang.
Vorausgesetzt werden \textbf{Rogolan}-Kenntnisse bei mindestens einem Gruppenmitglied sowie ein gewisses Interesse an zwergischer Kultur und dem Wohlergehen des Bergkönigreichs.
Orks und Echsenmenschen sind sicher ungeeignet, weil das \fkv ihresgleichen nicht in ihrer Stadt dulden würden.
Goblins und (Halb-)Elfen werden mit Anfeindungen und Unverständnis zu kämpfen haben.

Grundsätzlich spricht nichts dagegen, das Abenteuer auch mit einer rein zwergischen Gruppe zu spielen.
In diesem Fall liegt eine Herausforderung darin,  das Besondere an zwergischer Kultur für ihre Mitspielenden rüberzubringen,
während es für ihre Figuren selbstverständlich ist.
Sie können jedoch davon ausgehen, dass das \fkv ein durchaus eigenes Völkchen ist.
Viele Details dürften daher auch für  Vettern und Cousinen aus dem Eisenwald oder Amboss-Gebirge erwähnenswert sein.

\subsection*{Themen, Motive und Verbindungen}
Das Abenteuer ist kompakt, ermöglicht  ihrer Gruppe jedoch den Einstieg in eine zwergische Kultur
-- im wahrsten Sinne des Wortes, geht es doch hinab in Abschnitte einer Binge,
die das \fkv selbst nur noch in  Ausnahmefällen betritt.

Legen sie daher Wert darauf, diese Kultur auch vorzustellen. Das Abenteuer bedient hier insbesondere
\begin{itemize}
	\item das andere Zeitgefühl der langlebigen Spezies,
	\item die Bindung an die Tradition und Misstrauen gegenüber anderen Völkern,
	\item der hohe Respekt vor dem Alter und
	\item die an Sturheit grenzende Konsequenz.
\end{itemize}

Sie können hier  Spuren zur untergegangenen \textbf{Binge Umrazim} (\emph{KK:173}) legen,
falls ihre Gruppe im späteren Spiel nach diesem großen Geheimnis der Kinder Angroschs suchen soll.
In diesem Sinne habe ich es an \emph{Wohin der Schatten fällt} von Kathrin Lieb (in: \emph{Ehrenhändel} (FanPro, 2006)) angeschlossen,
wo ebenfalls Hinweise auf \textbf{Umrazim} zu finden sind.

Ebenso ruht unter \textbf{Finsterkoppen} -- tiefer, als dieses Abenteuer führt -- bis 1011 BF der \textbf{Salamanderstein}, der als Zeichen des \textbf{Saljeth-Paktes} Symbol des geeinten Kampfs von Kindern Angroschs und Elfen gegen die Orkgefahr darstellt.
Die Suche nach diesem Artefakt könnte die Gruppe nach \textbf{Finsterkoppen} führen.
Genauso wäre es möglich, einige Jahre später im Rahmen dieser Suche zurückzukehren.
Und hat der Salamanderstein nicht vielleicht etwas mit dem \textbf{Stein der Simia} (\emph{Brogars Blut}, Fanpro 1998), dem \textbf{Feuer Ingras} (\emph{Feuerbringer}, Ulisses 2013) oder \textbf{Angrosch ka Broschrardosch} (\emph{KK:111}) zu tun?

Weitere verlassene Bauten der Finsterkamm-Zwerge kommen in den Abenteuern \emph{Odem der Kälte} (Fanpro, 2010) sowie \emph{Das Geheimnis des Drachenritters} (Ulisses, 2019) vor.
\spaltenende
\begin{center}
	\zeichnung[0.75\textwidth]{Drache.jpg}
	\end{center}

\neueseite

\section{Erster\,Akt: Der Tiefe Rat}
\spaltenanfang
\info{\textbf{Ziel der Gruppe:} Das \fkv über die Gefahr informieren und den Auftrag erhalten, das Szepter zu suchen.

\smallskip

	\textbf{Vorschläge für längeres Spiel:}
	\begin{enumerate}
		\item Finsterkoppen im unwegsamen Finsterkamm zu finden kann ein Abenteuer für sich sein.
		Mehr dazu \link{https://brauerei.ilaris-online.de/share/jD-nmdtQr4ZM}{hier} (work in progress).
\item Nehmen Sie sich Zeit, Finsterkoppen zu erkunden -- nutzen Sie die Ortsbeschreibung im Anhang. Der \textbf{Tiefe Rat} kommt erst in einigen Tagen zusammen, zuvor kann die Gruppe gemeinsam ein zünftiges Pilzbier bei \textbf{Schwarzbart} trinken,
	von \textbf{Muragolosch} auf offener Straße ausgeschimpft werden,
	bei \textbf{Gandrasch} eine besondere Armbrust zu bestellen versuchen oder \textbf{Turgol} nach Geheimnissen der Geoden fragen --
	Bitten, die jetzt vielleicht abgeschlagen und nach dem dritten Akt erfüllt werden.
		\end{enumerate}
	\smallskip
	\textbf{Vorschläge für kürzeres Spiel:}
		\begin{enumerate}
		\item Lassen Sie die Gruppe statt der in der Ratsversammlung auf S.\,\pageref{idee} vorgesehenen Recherche noch in der Ratsversammlung eine \emph{Gebräuche-Probe (Zwerge, 24)} würfeln.
		Misslingt sie, geschieht nichts schlimmes, sondern \textbf{Bonderik} bringt das Wissen um das Szepter ein und wird von Inradon \textbf{Xermosch} bestätigt.
	\item	Auf dem \textbf{N\^orrnstieg} wird die Gruppe von einer zwergischen Schar mit vorgehaltenen Armbrüsten überfallen.
	Sie sollen eine Schatz -- das Szepter -- aus einer Binge bergen, die das \fkv selbst nicht mehr betreten will.
\end{enumerate}
}

\subsection{Finsterkoppen}
Finsterkoppen ist die Hauptstadt des Finsterkamm-Volkes.
Der Begriff \enquote{Stadt} trifft es nicht ganz: zumindest unterirdisch hat Finsterkoppen keine klare Grenze.
Die Wohnhöhlen, Minen und Bingen ziehen sich weit unter dem Gebirge hindurch.
Jede Sippe dürfte ein anderes Verständnis dafür haben, was davon noch zu Finsterkoppen gehört.

Vier Besonderheiten kommen jedoch auf engem Raum zusammen und machen deutlich, dass hier das politische und geistliche Zentrum des Bergkönigreichs liegt:
\begin{itemize}
	\item ein teilweise aufgegebenes und verschlossenes Bergwerk, das den Aufzeichnungen zufolge der Ausgangspunkt der zwergischen Besiedelung des Finsterkamms war (es wird im \emph{zweiten Akt} näher erkundet)
	\item der damit verbundene, zentrale \textbf{Angrosch}-Tempel (s. S.\,\pageref{angrosch})
	\item die Versammlungshalle/Halle des \textbf{Tiefen Rats} (s.\,u.)
	\item (für eine zwergische Siedlung)  ungewöhnlich viele oberirdische Gebäude, die auch den endgültigen Sieg über Horndrachen und Orks verdeutlichen,
	die in anderen Ecken des Gebirges durchaus noch Gefahren darstellen.
\end{itemize}

In diesem Abschnitt sind nur die zwei Schauplätze beschrieben, an denen die Handlung spielt: Die Versammlungshalle (in der der \textbf{Tiefe Rat} tagt) und der Angrosch-Tempel (durch den die Gruppe die Binge betreten kann). Weitere Details zu Finsterkoppen als Abenteuerschauplatz finden Sie im Anhang mitgeliefert.

\kasten{Großlinge würden vielleicht von einem \emph{Hohen Rat} sprechen -- deutliches Zeichen, dass die Sonne ihnen das Haupt verbrannt hat.
	Selbstverständlich ist es der \textbf{\emph{Tiefe} Rat}, was die \emph{tiefe} Weisheit seiner Mitglieder und Bedeutung seiner Beschlüsse anzeigt!}

\vfill

\subsubsection{Die Versammlungshalle}

\vorlesen{\label{halle}
Die Versammlungshalle verbindet das große Tor mit dem Angroschtempel. Vier weitere Gänge führen seitlich tiefer in den Berg.
Die Halle ist für zwergische Bedürfnisse wahrlich großzügig bemessen -- bei wichtigen religiösen Zeremonien finden mehrere Tausend hier Platz.
Im Verteidigungsfall können große Geschütze aufgestellt und auf den Gang gerichtet werden.
Häufiger dient sie zu repräsentativen Zwecken wie dem Empfang hohen Besuchs anderer Völker -- und den regelmäßigen Sitzungen des \textbf{Tiefen Rat}s.\\
In zwei Reihen sind je vier Podeste angeordnet, ein neuntes -- das des \textbf{Bergkönig}s -- an ihrer Stirnseite.
}

Die Podeste haben auch außerhalb der Ratssitzungen ihren Sinn.
Hitze aus dem Schlot des Berges wird nämlich durch ein sinnreiches Röhrensystem auch unter den Boden der Halle geleitet.
In etwa ein Spann großen Öffnungen auf den Podesten findet dieses System Ausgang, so dass sich einerseits kein Überdruck aufbaut,
wenn \enquote{Väterchen Angrosch mal kräftiger heizt} und andererseits solche spontanen Entladungen  nicht entlang von Laufwegen auftreten.

Regelmäßig flimmert daher die Luft über den Podesten vor Hitze, seltener treten schweflig riechende Wolken aus,
noch seltener Gase, die Traumbilder verursachen. Erst bei einer Hand voll Gelegenheiten in tausendjährigen Geschichte dieser Halle ist so ein Austritt mit einer Ratssitzung zusammengefallen;
das \fkv interpretiert diesen Einfluss auf die Ratsmitglieder als direkte Inspiration durch Angrosch.

\subsubsection{Der Angroschtempel}\label{angrosch}
\vorlesen{%
Im gesamten Tempel ist es \emph{heiß} \emph{(Ilaris:35)}, da stets mehrere Feuer und Öfen brennen.
Seine Grundfläche entspricht etwa der der Versammlungshalle, doch gibt es hier mehr Nischen mit \textbf{Altäre}n -- oft zugleich Werkbänke oder Ambosse.
Die Wände sind mit Angram-Runen bedeckt, die die Geschichte der Kinder Angroschs im Allgemeinen und des \fkv im Besonderen im Lichte des Angrosch-Glaubens auslegen.\\
Den Platz der heiligen Flamme nimmt in diesem Tempel ein sechs Spann breiter \textbf{Schlot} ein, der direkt senkrecht in das vulkanische Herz der Berge führt.
Hier hinein werden Opfergaben und die Toten aus dem \fkv geworfen.
Die Hitze aus der Tiefe ist in zwei Schritt Umkreis deutlich spürbar (\emph{Kh\^omglut}, \emph{Ilaris:35}).
Hier treten mehrmals pro Jahr  und in höherer Konzentration als in der Versammlungshalle Gase aus, die bei empfindlichen Zwergen und Menschen Atemnot,
bei empfänglichen (\emph{Vorteil: Prophezeien (Ilaris:30)} zudem Visionen auslösen können.
\\
Hinter dem Schlot führen einige Stufen hinauf zum großen Altar-Amboss. Auf dessen Rückseite setzen sich die Stufen zu einer langen \textbf{Treppe} hinab in die Dunkelheit fort.
}

Diese Treppe endet vor einer kleinen Halle, die außer einigen alten Kohlebecken nichts enthält.
Ein stets verschlossenes \textbf{goldenes Portal} führt von hier aus in den aufgegebenen Teil der Finsterkopp-Binge.

\zeichnung[0.3\textwidth]{hammer.png}

\subsection{Zur Ratsversammlung}
\subsubsection{Der Tiefe Rat}
Der \textbf{Tiefe Rat} trifft jene Entscheidungen, die das  gesamte Bergkönigreich angehen.
Er setzt sich aus den Häuptern der acht bedeutendsten Sippen zusammen.
Mit dieser Zahl erinnert das traditionsbewusste \fkv an die acht Stammväter und acht Stammütter (\emph{AK:8}) --
seit Beginn der Siedlung sind somit immer acht Sippen vertreten gewesen, auch wenn sich alle paar Jahrhunderte ändert, welche acht.


Diese acht wählen aus ihrer Mitte den \textbf{Bergkönig} oder die \textbf{Bergkönigin}, für die dann ein weiteres Mitglied aus derselben Sippe nachrückt.
Als Anführer der Golgrasch-Sippe tritt an Stelle von Bergkönig \textbf{Gerambalosch Sohn des Gengram} somit dessen Sohn
\textbf{Garbalon} (tatkräftig, schlägt darum gern die Hände zusammen) auf.

\neuespalte

Die übrigen Sippen sind vertreten wie folgt:
\begin{itemize}
	\label{rat}
\item \textbf{Bonderik} (schwarzhaariger Zwerg mit knarzender Stimme, legt die Fingerspitzen aneinander)
\item \textbf{Parsezel Sohn des Pemoin} (Zwerg, spricht aufgeregt, wirft an passenden und unpassenden Stellen ein \enquote{oder?} ein)
\item \textbf{Rodrik} (ungewöhnlich dicker Zwerg, der gern in ein- und zwei-Wort-Sätzen spricht, klopft sich aufs Knie, wenn er fertig ist)
\item \textbf{Barselok} (weißhaarige Zwergin, die jeden Satz mit einem gesummten mmmh beginnt)
\item \textbf{Fendor} (außerordentlich gutaussehender Zwerg, unterstreicht seine Worte mit Fingerspiel)
\item \textbf{Drumbal} (Zwerg, der beim Sprechen kaum die Zähne auseinander nimmt)
\item \textbf{Borsok}  (von vielen Kämpfen vernarbte Zwergin)
\end{itemize}

Der Geode \textbf{Turgol} (Kapuze stets tief ins Gesicht gezogen, denkt über jedes Wort dreimal nach) ist von der Hand voll \textbf{Geoden}, die es im Finsterkamm gibt, als ihr Sprecher anerkannt.
Am Rat nimmt er jedoch als Berater des Bergkönigs teil und hat selbst keine Stimme.

Der Hochgeweihte (\enquote{Inradon}) \textbf{Xermosch} des örtlichen Angrosch-Tempels 
(kahlköpfig, bezieht sich in jedem zweiten Satz auf Angrosch) ist offiziell ebenfalls nur beratend bei den Sitzungen des tiefen Rates.
Im Gegensatz zu den abergläubisch gefürchteten Geoden kann er aber voraussetzen, dass seinen Worten Gewicht beigemessen wird.

\info{Vermutlich wird es für Ihre Gruppe -- und für Sie -- zu unübersichtlich, wenn sie in der Versammlung alle Ratsmitglieder inklusive den beiden Beratern als Sprechrollen führen. Nutzen Sie \textbf{Bonderik} als klugen Mahner, \textbf{Garbalon} als sympathischen Aktionisten,  \textbf{Borsok}, um der Gruppe kontra zu geben und \textbf{Xermosch} als Wissenshüter.\\Dennoch ist es nicht schlecht, weitere Rollen in der Hinterhand zu haben, falls auch in Ihrer Gruppe unterschiedliche Positionen eröffnet werden und die Verhandlungen sich im Spiel in die Länge ziehen.}

\kasten{\textbf{Bergkönig oder Hochkönig?}
	
	Das ständige Amt, das die Menschen als \enquote{\textbf{Bergkönig}} bezeichnen, ist nicht mit den  Befugnissen ausgestattet, die man gemeinhin für einen König erwartet.
	Es handelt sich eher um einen obersten Friedensrichter des jeweiligen zwergischen Volkes.
	Gerade im \fkv\ ist seine formale Macht begrenzt -- er ist lediglich Vorsitzender des Tiefen Rates ohne eigenes Stimmrecht.
	Nur bei Stimmengleichstand obliegt es ihm, den Ausschlag zu geben.
	
	In Zeiten äußerer Gefahr oder großer Not wählen die Zwerge dagegen \textbf{Hochkönig}e --
	formal über alle Kinder Angroschs, wobei gerade das \fkv\ sich schon öfter nicht die Mühe gemacht hat,
	andere Völker in die Wahl einzubeziehen oder die Anerkennung des Hochkönigs von diesen zu verlangen.
	
	Vom Hochkönig wird gemeinhin erwartet, dass er sein Amt aufgibt, wenn die Gefahr vorüber ist.
}

\subsubsection{Der Ablauf der Sitzung}

\textbf{Auftritt des Bergkönigs}

\label{ratssitzung}
\vorlesen{%
Der Bergkönig wird von vier kräftigen Angehörigen seiner Sippe auf einem Stuhl mit hoher und reich verzierter Lehne hereingetragen.
Sie setzen ihre Last ab, ziehen die Tragestangen heraus und ziehen sich unauffällig zurück.

\textbf{Gerambolosch}s mit Hermelinfell gesäumte Robe aus schwerem roten Samt wallt über dem Kettenhemd, das mit breiten goldenen Armreifen am Handgelenk
 abschließt.
Der prächtige weiße Bart ruckt auf der Brust des Bergkönigs hin und her,
 den Bewegungen seines herabgesunkenen Kopfs folgend, wie man es von Schlafenden kennt. Die Lider sind halb über die Augen gesunken. Deren untere Hälfte wirkt milchig.
\\
Nachdem der Stuhl abgesetzt wurde, lässt \textbf{Gerambolosch} keine Regung mehr erkennen.
}


\textbf{Der Bericht}

	Statt dessen ergreift \textbf{Xermosch} das Wort (selbstverständlich auf Rogolan):

\label{gefahr}

\vorlesen{%

	
\enquote{%
	Geschätzter König Gerambolosch, hochverehrte Mütterchen und Väterchen: Angrosch ist mein Zeuge, viele Fragen treiben den Tiefen Rat heute um.
Tauwetter steht uns bevor und wie jedes Jahr müsst Ihr beschließen, wann welche Schleuse geöffnet und welche Weide bewässert werden soll.
Angroschs Tempel in Hilltorp erbittet einen weiteren Baumeister für die Reparatur eines Amboss-Sockels in der Tempelhalle.
	
	Zu allererst würde ich jedoch gern einen Punkt aus der Welt außerhalb unseres von Angroschs gesegneten Bergkönigreichs aufrufen:
	Die Fremden, die Ihr hier vor euch seht, berichten von einer -- weiß Angrosch! -- großen Gefahr.
	Wir haben sie darum gebeten, persönlich vor Euch zu erscheinen, damit Ihr den Bericht aus ihrem eigenen Munde hört.
	So keine Gegenrede besteht, wollten wir ihnen unter Angroschs Augen das Wort erteilen.}
}

\textbf{Bonderik} erhebt eben solche:

\vorlesen{%
\enquote{Hochverehrter Vater Xermosch, Eile mit Weile und alles, wie es Angrosch gefügt! Wie Ihr selbst sagt, sind es Fremde.
Einige von ihnen nicht einmal Angroschim. Ich würde doch gern zunächst wissen, um wen es sich handelt und ob es vertrauenswürdige Leute von Stand und Ansehen sind, bevor ich ihnen mein Ohr leihe.}
}

\info{Die Gruppe hat nun Gelegenheit, von ihrer Person und ihren bisherigen Taten zu berichten.
Menschlicher Adel und akademische Erfolge -- selbst ein Krieger*innenbrief -- zählen wenig.
Siegreiche (oder doch zumindest mutige) Kämpfe gegen Drachen, Oger und Orks wissen die Ratsmitglieder dagegen zu schätzen.}

\info{%
Legen Sie die Latte für Vertrauenswürdigkeit nicht zu hoch. Der anschließende Konflikt soll sich am Umgang mit der Gefahr entzünden, nicht die Wahrheit des Berichts in Zweifel gezogen werden. Anschließend werden sie daher gebeten, die drohende Gefahr darzustellen. %Nutzen Sie die Möglichkeit, durch gelegentliche kurze Einwürfe die übrigen Ratsmitglieder vorzustellen.
In der Diskussion sollte deutlich werden, dass das \fkv gewohnt ist, sich gegen alle Gefahren in den Stollen und Höhlen seines Bergkönigreichs zu verschanzen.
Verantwortung für die Reiche der Menschen oder gar Elfen übernimmt es nicht.
}

Während des Berichts zeigt der Bergkönig weiterhin kein Zeichen des Zuhörens oder des Erwachens. Wer bewusst darauf achtet, kann immerhin ab und an Heben und Senken der Brust erkennen.
Der Bergkönig lebt.

\probenkasten[
bild=wahrnehmung,
zusammenarbeit=nein,
gruppenprobe=nein,
pw=12,
farbe=gruen,
erfolg={},
misserfolg={}
]{Menschenkenntnis}{%
 Die übrigen Teilnehmenden scheinen \textbf{Gerambolosch}s Zustand vollständig zu ignorieren. \\
 \textbf{16:} Sie unternehmen sogar bewusste Anstrengung, ihn nicht anzusprechen oder direkt anzusehen.\\
 \textbf{20:} Der Zustand scheint ihnen unangenehm oder peinlich zu sein.\\
 \textbf{24:} Offensichtlich befürchten die Teilnehmenden, dass \textbf{Gerambolosch} in den Zustand der Vergreisung eingetreten ist --
 jenen bei manchen Zwergen auftretenden rapiden Verfall von Körper und Geist, der an Stelle der hoch geschätzen Weisheit des Alters tritt und vom Tod gefolgt wird (\emph{DZA:20}). Dieses ruhmlose Ende wünscht man seinem ärgsten Feind nicht -- die Ratsmitglieder wollen nicht darüber nachdenken.\\
 \textbf{28:} Gerade die milchigen Augen des Bergkönigs und eine schwer zu bemerkende Verfärbung der Fingernägel sprechen aber eher für eine Krankheit oder eine Vergiftung.%
 }

\vfill

\neueseite
\textbf{Das Ziel}

\info{Natürlich müssen Sie die folgende Diskussion an die von Ihnen gewählte Gefahr für das \fkv anpassen. %ist eher beispielhaft ausführlich dargestellt.
	Geben Sie Ihrer Gruppe die Gelegenheit für Einwürfe, die
	-- solange sie ihren Gaststatus beachten und dem Rat mit Achtung begegnen -- auch argumentativ aufgegriffen werden.}



Als erstes spricht \textbf{Garbalon}: 
\vorlesen{%
\enquote{%
	Das klingt wirklich ernst. Ich für meinen Teil bin froh, dass wir so früh davon erfahren.
	Wir können nicht nichts tun! Wie mein Vater, einer der ältesten und gewiss der weiseste Angroscho unter dem Berg mehr als einmal gesagt hat, war gestern der beste Tag, ein Werk zu beginnen und ist heute der zweitbeste. 	Also sage ich: Essen anheizen, Waffen schmieden, Vorräte anlegen, Befestigungen verstärken.%
}
}

\textbf{Bonderik} wirft ein:
\vorlesen{
\enquote{Das ist schneller gesagt als getan. Ich bin mir nicht sicher, ehrwürdiger Garbalon, ob ich auf eurer Seite stehe.
Und selbst wenn: Wie viele Waffen genau, wie viele Vorräte, welche Befestigungen? Wir haben die besten Bauleute und Schmieden.
Aber wir sind ein kleines Volk und jede Stunde kann nur einmal gearbeitet werden.}
}
%Borsok erklärt: \enquote{Wenn wir noch einige Monate Zeit haben, sollten wir an Silber schürfen, was wir können.
%Für das Silber können die Menschen uns mehr essbare Vorräte verkaufen, als wir selbst in dieser Zeit anbauen könnten.}

%Bonderik fragt: \enquote{Ich will gewiss die Arbeit eurer Sippe nicht schmälern, die ergiebige Silberadern wohl zu nutzen weiß.
%Was aber, wenn das Erntejahr in der Ebene schlecht ausfällt? Oder wenn die Großlinge nicht mehr verkaufen wollen, weil sie ebenfalls einen Krieg mit den Orken fürchten?}

So geht es eine Weile hin und her, bis \textbf{Parsezel} fragt: 
\vorlesen{
\enquote{Vielleicht ist der Tiefe Rat nicht der richtige Ort für diese Frage, oder?
	Wenn wir alle glauben -- oder? -- dass uns wirklich dieses Schicksal bevorsteht, muss vielleicht ein anderer -- oder?  -- Weg gefunden werden.
	In Zeiten großer Gefahr hat unser Volk immer den Hochkönig der Zwerge und Zwerginnen gewählt \dots\ oder?}\\
Für einige Atemzüge herrscht peinliche Stille.\\
\enquote{Wohl wahr, bei Angrosch}, sagt Inradon Xermosch schließlich langsam. Wie um Zeit zu gewinnen, erklärt er die Gesetze, die den meisten Anwesenden klar sind:
\enquote{Ein Hochkönig. Die Wahl eines Hochkönigs vorzuschlagen ist vor Angroschs Augen das Vorrecht des Bergkönigs. Er ist in diesem Fall der erste Kandidat.
Falls er auf die Kandidatur verzichtet, obliegt es ihm, ein anderes von Angroschs Kindern zur Wahl zu benennen.} Seine Stimme gerät ins Zittern:
\enquote{Erst danach können andere Kanditaturen erklärt werden und die Wahl erfolgt durch den Tiefen Rat.}
}

\neuespalte

\textbf{Garbalon} ist rot angelaufen. Er erklärt:

\vorlesen{%
 \enquote{Mein Vater wäre gewiss ein guter Hochkönig. Seiner Weisheit kommt kaum jemand in unserem Volk gleich. Seine Kampferfahrung \dots}
 Er bricht leicht verzweifelt ab. Dann fast er seinen Mut zusammen und wendet sich an Gerambolosch:
 \enquote{Geliebter, ehrwürdigster Vater! Befürwortet Ihr die Hochkönigswahl? Stellt Ihr Euch zur Verfügung?}
 
Doch es scheint, als würde der Bergkönig ihn nicht hören.}

\info{
Eine Pattsituation: Die Mehrheit im Rat könnte sich auf die Wahl einigen. Auch mögliche Kandidaten gibt es.
Niemand wird jedoch vorschlagen, die Tradition zu brechen und im Wahlverfahren den Bergkönig zu übergehen.

Die Gruppe kann sich mit einem solchen Vorschlag eine harsche Zurechtweisung einhandeln.
Auch das Angebot, den König -- gar mit magischen Mitteln -- zu untersuchen, weckt Widerstände.}

Die Ratsmitglieder sind bereit, die Diskussion mit zwergischer Gründlichkeit fortzuführen.
Argumentativ bewegen sich dabei im Kreis.
Nach -- für Menschen endlosen -- Stunden vertagt man sich ergebnislos.

\kasten{\textbf{Bonderiks Plan}\\
\textbf{Bonderik} ist einer der klügsten Zwerge im Rat und leidet darunter, dass dem alten Bergkönig regelmäßig mehr Gehör geschenkt wird als ihm. Daher lässt er diesen seit einiger Zeit durch \textbf{Turgol}s Schüler \textbf{Muragolosch} vergiften.
Der Vorschlag einer Hochkönigswahl eröffnet ihm nun neue Möglichkeiten.
Er würde sich nicht als Kandidaten benennen, um seine Position nicht zu schwächen.
Er weiß jedoch, dass das \textbf{Szepter der Stärke}, denjenigen wählen wird,
der ihm etwas Rattengold beigibt und dass dieses Material in der Finsterkopp-Binge zu finden ist.
}

\subsubsection{Den Ausweg finden}
Im Anschluss kommt \textbf{Garbalon} auf euch zu:
\vorlesen{
\enquote{Ihr habt recht getan, dass Ihr uns die Kunde brachtet. Ich wünschte wirklich, wir wüssten schon, was als nächstes zu tun ist. Aber es scheint, als sei heute kein Weg voran zu finden.}

\textbf{Bonderik} mischt sich ein:
\enquote{Ich bin mir da nicht so sicher. Ich meine mich zu erinnern, dass es zu Zeiten unserer Vorväter auch andere Wege gab, einen Hochkönig zu bestimmen.
Müsste das nicht im Tempel verzeichnet sein? Mein Angram ist leider seit meiner Jugend nicht besser geworden \dots}}

\spaltenende
\label{idee}
\probenkasten[%
bild=gebraeuche,
gruppenprobe=gut,
zusammenarbeit=ja,
farbe=gruen,
pw=24,
misserfolg={	\textbf{Bonderik} wird Inradon \textbf{Xermosch} -- der nicht selbst auf die Idee kommt -- bitten, die \textbf{Angram}-Runen zu studieren.
	Damit geht er das Risiko ein, dass auch die Möglichkeit der Manipulation zur Sprache kommt.
	Tatsächlich hält der Inradon diese Möglichkeit jedoch für eher theoretisch und wird sie nicht erwähnen.},
erfolg={In den alten Tagen gab es noch ein zweites Verfahren der Hochkönigwahl:
	Der Inradon schmiedete im Bedarfsfall das \textbf{Szepter der Stärke} neu. Danach machte es Angroschs Willen offenbar und bestimmte den Hochkönig.\\
	\textbf{28:} Das Szepter wurde in einem aufgegebenen Teil der Mine eingelagert. Dort müsste es eigentlich bis heute liegen.\\
	\textbf{32:} Der Brauch wurde aufgegeben, nachdem eine fremde Gottheit einen Weg gefunden hatte, die Wahl zu unterwandern.\\
	\textbf{36:} Es handelte sich um den \textbf{Namenlosen}, der durch seine Gläubigen etwas \textbf{Rattengold} unter die Schmiedematerialien geben ließ.\\
}
]{Ermittlung (Ilaris:69): Gebräuche (Zwerge)}{
Mit dieser Information kann die Gruppe eine \emph{Ermittlung} anstellen, die den  größeren Teil eines Tages in Anspruch nimmt.
Für den nächsten Tag wird der \textbf{Tiefe Rat} erneut zusammengerufen.
}

\spaltenanfang

\textbf{Barselok} setzt an:
\vorlesen{
\enquote{%
Mmmhabt Dank dafür, an die Tradition zu erinnern.
Mmmhaber wenn das Zepter in der alten Binge liegt, ist es unerreichbar.
Mmmhdiese Binge nicht mehr zu betreten, ist ebenfalls Teil unserer Tradition.}
\\
\enquote{In der Tat}, erklärt der Inradon, \enquote{der Schlüssel zum goldenen Portal wird in unserem Tempel unter Angroschs Augen aufbewahrt, aber nicht mehr genutzt.}
}

Es handelt sich nicht um ein echtes Tabu oder ein Gesetz, sondern um Tradition.
Niemand aus dem \fkv betritt die Binge, weil niemand es tut.
Es werden auf Nachfrage sogar unterschiedliche Gründe genannt, warum die Binge aufgegeben wurde:
\begin{itemize}
\item Dass die Bodenschätze zur Neige gegangen seien,
\item dass \textbf{Angrosch} dem \fkv auferlegt habe, dort Geheimnisse zu bewahren,
\item dass \textbf{Angrosch} die Mine verflucht habe, weil man in einer Zeit der Verwirrung auch Elfen als Gäste dort bewirtet hätte,
\item \dots
\end{itemize}
\info{
Vermutlich wird die Gruppe vorschlagen, sich um dieses Problem zu kümmern.
Der Tiefe Rat steht dem aus Gründen der Tradition kritisch gegenüber.
Damit ist ein Rededuell (\emph{Ilaris:55}) mit DG~2 begonnen:
}
\spaltenende

\probenkasten[bild=beeinflussung, zusammenarbeit=ja, gruppenprobe=gut, pw=s.u., detailgrad=2, vergleichend=ja, anzahl=1,
erfolg={Die Gruppe wird beauftragt, in die Binge hinabzusteigen.
	Inradon \textbf{Xermosch} geleitet sie vor das goldene Tor und überreicht Ihnen den Schlüssel:
	
	\vorlesen{
		\enquote{Ihr werdet das Szepter auf der dritten Ebene finden --
			wenn ihr dem Schacht bis zu den letzten Plätzen gefolgt seid, wo wir in Angroschs Namen bis zuletzt Erz abbauten,
			biegt links ab und folgt dem Duft nach Magma.
			
			Aber gebt acht, dass ihr nicht den Zorn Angroschs auf euch zieht.
			Er wird jeden zerschmettern, der den Zugang sucht, ohne das Andenken der Vorfahren zu ehren.
			Denkt deswegen daran, im alten Tempel direkt nach dem goldenen Tor an Angroschs Altar zu beten,
			um die Weisheit der richtigen Wahl.
			
			Dann mögt ihr das Szepter der Stärke  bergen, auf dass es neu geschmiedet werde.}
}},
misserfolg={Verliert die Gruppe das Duell, wird sie wahr\-schein\-lich dennoch in die Binge wollen.
	Dies sollten Sie auch ermöglichen.
	
	\info{Wenn Ihrer Gruppe die zwergischen Gebräuche zu bedeutsam erscheinen, um sich darüber hinwegzusetzen, schieben Sie ein vertrauliches Gespräch mit \textbf{Borsok} ein, die -- für die Gruppe überraschend -- mehr oder weniger deutlich zu verstehen gibt, dass diese Lösung für alle das Beste wäre, weil gerade durch den Status der Gruppe als Außenseiter der Verstoß nicht so schwer wöge.}
	
	Im \emph{zweiten Akt} wird die Gruppe es dann schwerer haben:
	Ohne die Hinweise des Hohepriesters werden sie am Angrosch-Altar nicht ahnen, dass sie später ein Rätsel zu lösen haben.
	
	Im \emph{dritten Akt} werden viele Ratsmitglieder in ihnen zwielichtige oder ehrlose Gestalten sehen.
}
]
{Rededuell}{Für Menschen wird \emph{ungewohnte Umgebung} angenommen.
\emph{Überreden} und \emph{Rhetorik} sind \emph{angemessene} Talente, ebenso \emph{Einschüchtern} --
falls die Gruppe die äußere Bedrohung -- also beispielsweise die Gier und die Kampfkraft der Orks -- und nicht etwa die eigene ins Feld führt.
Andernfalls wäre es ebenso \emph{unsinnig} wie \emph{Betören}.\\
Für den tiefen Rat gelten folgende Werte: \emph{Willenskraft} 14, \emph{Menschenkenntnis} 10, MU 12, KL 10.
}

\spaltenanfang

Hier drei Ideen für alternative Wege hinab -- ergänzen Sie bei Bedarf, was zu Ihrer Gruppe passt, bis etwas gelingt:
\probe{heimlichkeit}{DG\,2 (20): Untertauchen, Stehlen}{Der Schlüssel wird in den Tempelräumen verwahrt.
Er ist nicht öffentlich zugänglich, aber auch nicht bewacht und kann mit etwas Geschick zunächst unbemerkt ausgeliehen werden.
}

\probe{wahrnehmung}{DG\,2 (20): Men\-schenkenntnis, Einschüch\-tern}{
Ebenso könnte ein jüngerer Geweihter oder Tempeldiener überzeugt werden, den Schlüssel zu besorgen.\\
}
Hierbei führt der Misserfolg zu einem Betretungsverbot des Tempels (wenn auch nur für die nächsten 3x3 Jahrzehnte).

\bigskip

Das Schloss des goldenen Tors ist echte Zwergenarbeit und nur mit enormem Aufwand zu knacken. Dieses Tor wurde aber erst später in der Nutzungsgeschichte errichtet. 
\probe{koerperkraft}{DG\,2 (20): Körperkraft, Mechanik}{Ein vor Jahrhunderten vermauerter, weniger repräsentativer Eingang ist hinter einem der Kohlebecken zu entdecken. Er lässt sich mit einfachem Werkzeug
freilegen und aus den Angeln heben.
}
Misslingt die Probe, führen diese groben Arbeiten zu einem Steinschlag (4W6 TP für die würfelnde Figur).
\begin{center}
\zeichnung[0.18\textwidth]{Gruppenprobe.jpg}
\end{center}

\spaltenende

\neueseite

\section{Zweiter\,Akt: Die Finsterkopp-\allowbreak Binge}
\spaltenanfang
\info{\textbf{Ziel der Gruppe:} Das Szepter finden und Hinweise auf den Saboteur erhalten.
	\smallskip
	
	\textbf{Vorschläge für längeres Spiel:}
	\begin{enumerate}
		\item Durch die Gänge der Wühlschrate sind neben der Riesenamöbe auch \link{https://ilaris-online.de/app/kreatur/92}{Höhlenspinnen} (\emph{Ilaris:101}) und \link{https://ilaris-online.de/app/kreatur/81}{Gruftasseln} eingedrungen.
		\item Der Bosnickel hat Lust auf Gesellschaft und einen zünftigen Rätselwettstreit und erzählt erst danach etwas von seinen Umtrieben.
		\item Der Fund des Szepters weckt den Geist von einem von \textbf{Geramboloschs} Vorgängern.
		Die Erscheinung lockt die Gruppe zu einem zugemauerten Eingang, doch wird dieser aufgebrochen, finden sich dahinter bloß einige seiner Gefolgsleute, die sich untot erheben -- schwerfällige, aber auch schwer gerüstete und furchterregende Gestalten.
	\end{enumerate}
\smallskip
	\textbf{Vorschläge für kürzeres Spiel:}
	\begin{enumerate}
	\item Streichen Sie die Wühlschrate.
	\item Verzichten Sie auf die Begegnung mit dem Bosnickel und seien sie stattdessen großzügig mit den Informationen beim Auffinden des Rattengolds.
\end{enumerate}
}
\vorlesen{
Der reich verzierte, vierbärtige Goldschlüssel passt in das große Schloss des goldenen Portals. Nach jeweils einer Viertelumdrehung ist Klicken und Schaben zu hören -- offensichtlich öffnen sich mehrere Riegel. Schließlich schwingt die Tür -- trotz ihres Alters und Gewichts beinahe lautlos -- nach außen auf. Solange die Tür offen steht, ist der Schlüssel beim besten Willen nicht zu entfernen, ohne ihn zu zerstören.}

\kasten{Da das Tor offen bleibt, kann sich \textbf{Bonderik} über die Tradition hinweg setzen und in einem unbeachteten Moment hinter der Gruppe in die Binge eindringen.}

Die Gruppe ist auf das \textbf{Licht} angewiesen, das sie mitbringt.
Zwar gibt es in fast allen Räumen Fackelhalter oder sogar Ölschalen, die über ausgeklügelte Rinnen verbunden sind, doch kein Öl dafür.
Vereinzelt hat sich Phosphorpilz ausgebreitet, so dass \emph{Sternenlicht (Ilaris:38)} herrscht.
%Rund um den Feuerschlot ist ab und an ein leichtes Glosen wahrnehmbar. Eine richtige Beleuchtung stellt dieses jedoch nicht dar.

Die gesamte Binge gilt als \emph{enger Raum (Ilaris:38)}.
Die meisten Gänge und viele der Räume haben nur 8 Spann Deckenhöhe.
Hier sind zwerg- oder wühlschratgroße Kämpfende gegenüber größeren stets \emph{vorteilhaft} positioniert.
Ausnahmen bilden einige der Werkstätten (für Luftzirkulation) und  Tempelräume (aus repräsentativen Gründen).

\subsection{Die erste Ebene}

\subsubsection{Verlassene Tempelräume}
Unmittelbar hinter dem Portal liegen Räume, die unschwer als älterer und aufgegebener \textbf{Tempel} identifiziert werden können,
beispielsweise an einem großen \textbf{Angrosch}-Relief, das einstmals mit großer Kunstfertigkeit gestaltet wurde, aber in der Tiefe verblieben ist.
Die Raumaufteilung mit ihren Nischen ähnelt dem neueren Tempel.
Die Werkzeuge und metallenen Amboss-Altäre sind jedoch mit umgezogen.

Nur ein steinerner Altar ist zurückgeblieben. Dahinter sind in die Wand acht Runen des \textbf{Rogolan} eingelassen, umgeben jeweils von acht Feldern, die manchmal ein Amboss-Symbol zeigen und manchmal nicht.
Der Name Angroschs -- \includegraphics[scale=0.4]{ngrsch.png} --
prangt darüber.
\kasten{Es handelt sich um die Anleitung für die \textbf{Rätseltür}, die den Zugang zum \textbf{Szepter} versperrt.}
\info{Zeigen Sie die Seiten im Anhang (ab S.\,\pageref{karten}) und händigen Sie sie aus, falls die Gruppe eine solche Bedeutung vermutet und sich entsprechende Notizen macht. 
	
	Andernfalls wird sie später noch einmal wiederkommen müssen.}
In den Tempelräumen findet sich auch die Fortsetzung des \textbf{Feuerschlot}s, der wegen der Hitze ohne magische oder karmale Hilfsmittel nicht zu beklettern ist.

\subsubsection{Fallen}
\info{In echt zwergischer Gründlichkeit hat das \fkv bei seinem Rückzug die Gänge mit einigen \textbf{Fallen} präpariert,
die Sie nach dramaturgischer Notwendigkeit im folgenden beliebig positionieren können.}

\textbf{Die Bolzenfalle}

\vorlesen{Als ihr um die Ecke biegt, spürst du, wie eine der Steinplatten unter deinen Füßen um eine halbe Fingerbreite nachgibt. Weiter geschieht nichts.
Die Platte lässt sich nicht mehr bewegen und unterscheidet sich nicht von den übrigen, mit denen der Boden hier belegt ist.}

\info{Hinter den Wänden und unter dem Boden sind Hohlräume, in denen ein System von Drähten verläuft.

Durch das Herabdrücken der Platte wird eine Bolzenfalle hinter der Wand am  gegenüberliegenden Ende des Ganges gespannt. Die auslösende Platten sind in der Mitte des Ganges ebenso unaufällig verlegt. Indem sie betätigt werden, wird das System auch zurückgesetzt und die erste Platte wieder angehoben.}

\zeichnung[0.35\textwidth]{Valeria.jpg}

Hat jemand in der Gruppe den \emph{Vorteil Zwergennase (Ilaris:30)} oder \emph{Gefahreninstinkt
(Ilaris:29)} ist eine Probe möglich (ohne diese Vorteile tritt die Wirkung der misslungenen Probe ein):

\spaltenende
\probenkasten[bild=wahrnehmung,
farbe=rot,
zusammenarbeit=nein,
gruppenprobe=gut,
vergleichend=nein,
pw=24,
erfolg={\vorlesen{Da ihr wisst, wonach ihr sucht, findet ihr bald Platten im Boden der Gangmitte, die ein wenig herausstehen. Vorsichtig übersteigt ihr sie. In der gegenüberliegenden Wand entdeckt ihr schließlich drei in Unebenheiten des Steins versteckte Löcher.}
Mit diesem Wissen ist es kein Problem, die Platten zu blockieren oder die Falle gefahrlos auszulösen.},
misserfolg={Zwei Figuren werden von Bolzen (2W6+4~TP) getroffen. \emph{Verteidigung} gegen diesen \emph{Fernkampfangriff (vgl. Ilaris:47)} ist nur möglich, wenn die Gruppe mit einer Falle rechnet und betont, wie vorsichtig (hinter den Schild geduckt, Schritt für Schritt \dots) sie sich durch den Gang bewegt.},
]{Wachsamkeit}{\mbox{} }

\spaltenanfang

\textbf{Die Grubenfalle}

\vorlesen{Auf dem Boden des Gangs liegt eine zerbrochene Lampe -- anscheinend menschlicher Bauweise -- und ein vom Alter mürber Beutel mit zerrissenem Tragriemen.}
In dem Beutel ist eine leere Glasflasche und ein Pergament. Bemerkenswerterweise handelt es sich um mit \textbf{Isdira}-Zeichen geschriebenes \textbf{Bosparano}. Der Unterschrift nach bestärkt Erzmagus \textbf{Asteratus Deliberas} \enquote{meine liebe Freundin und Kollegin}
ihre Forschung dazu zu vertiefen \enquote{ob nicht das kleine Volk und das schöne Volk} vor einigen Jahrhunderten  \enquote{Frieden und selbst ein Bündnis gegen die Geißel des Nordens} geschlossen hätten
und ob es Hinweise darauf gibt, wann dieses Bündnis gelöst wurde.

\spaltenende

\probenkasten[bild=wahrnehmung,
farbe=rot,
gruppenprobe=gut,
zusammenarbeit=ja,
pw=24,
erfolg={\vorlesen{Es handelt sich um eine einfache Wippfalle, bei der ein Teil des Bodens wegkippt und den unwillkommenen Besuch in eine Stachelgrube befördert. Da ihr sie entdeckt habt, könnt ihr euch gegenseitig so sichern, dass ihr die andere Seite problemlos  erreicht.}},
misserfolg={Bis zu drei Figuren (wenn die Gruppe eine Marschordnung festgelegt hat: die hintersten, ansonsten: die mit den wenigsten Wunden) legen eine \emph{Akrobatik-Probe (I, 24)} ab, bei deren Misslingen sie Bekanntschaft mit den Stacheln am Grund der Grube in Form von 3W6+3 TP schließen.
\vorlesen{	Bevor die Fackel erlöschen und Lampen zerbrechen zeigt ihr letztes Aufflackern die herumliegenden Knochen derer, die dieses Geschick zuvor ereilte.}
	Aus der Grube zu klettern dürfte nicht schwer sein, wenn die übrigen Gruppenmitglieder von oben den Bodenteil wieder ankippen.},
]
{Sinnenschärfe}{
Die folgende Untersuchung des Ganges
ist für Zwergennase oder Gefahreninstinkt jeweils -4 leichter.}

\spaltenanfang

Eine Ecke weiter findet sich dann ein Hebel an der Wand, der den Mechanismus deaktiviert.


\subsubsection{Verlassene Lagerräume}
Einige Truhen wurden wegen ihres schlechten Zustandes einfach zurückgelassen. 
Sie enthalten jedoch nichts mehr von Wert.
Daneben gibt es auch noch Felsnischen.
In einer davon liegt eine seit Jahrhunderten gespannte Bärenfalle \emph{(Ilaris:63)}.
Wer nach Beute sucht und eine \emph{Sinnenschärfe-Probe (III, 24, -4} durch jede Stufe \emph{Angepasst (Dunkelheit)}) misslingt, schnappt die Falle zu und Wundfieber (\emph{Ilaris:35}) droht.

Anschließend oder bei gelungener Probe wird ein verzauberter \textbf{Asthenil}-Ring gefunden (\emph{Ilaris:78}: \emph{ladungsbasiert, aufladbar}, eine Ladung, \biglittlecap{Elementarbann} mit \emph{Modifikation Gegenzauber} (\emph{Ilaris:126}), \emph{Erfolgswert} 20).
Das Artefakt hat einen wahrlich komplexen Auslöser:
Sobald ein Feuerzauber gegen die tragende Person gesprochen wird, wirkt der Gegenzauber.

\subsubsection{Verlassene Werkstätten}
In diesen Höhlen sind die Luftschächte deutlich häufiger. Doch liegt auch nach Jahrhunderten noch der Geschmack von Eisen in der Luft.
Nur an Spuren von Ruß und Abfällen lassen sich zuordnen, welche Höhlen für welches Handwerk genutzt wurde.
Einige einfache Werkzeuge wie Ketten, Hammer und Brechstangen wurden zurückgelassen, die verschleißbedingt die hohen Ansprüche der Kinder Angroschs nicht mehr erfüllten. Sie können aber durchaus noch benutzt werden, falls die Gruppe einen entsprechenden Bedarf haben sollte.

Etwas mehr Ausrüstung wurde einer ehemaligen Eisenhütte zurückgelassen, in der einst alchimistische Experimente durchgeführt wurden, die das \fkv bei seinem Auszug schon aufgegeben hatte.
Sie wäre als \emph{archaisches Labor} (\emph{Ilaris:64}) benutzbar.
Auch einige seltene mineralische und metallische Zutaten lassen sich mit etwas Mühe (und \textbf{Rogolan} (Schrift) III) finden.

\subsubsection{Verlassene Wohnhöhlen}
In diesem Teil hatten mehrere Sippen ihre Wohnstätten und lebten in jeweils mehreren Familien in guter Nachbarschaft.
Die Wohnhöhle einer Familie besteht jeweils aus einer kleinen Gemeinschafts-Halle, um die herum persönliche Schlafkammern angeordnet sind.
Mit einer Ausnahme -- siehe nächster Abschnitt -- sind sie alle restlos leer geräumt.
Da sie sich somit kaum unterscheiden, von jeder Gemeinschafts-Halle mehrere Schächte abgehen und diese wie die Ausgänge ebensooft nach oben wie nach unten oder ebenerdig verlaufen, kann man hier durchaus die Orientierung verlieren.

\neuespalte

\textbf{Der Bosnickel}

\vorlesen{Eine der Wohnhöhlen macht einen keineswegs verlassenen Eindruck.
Hier ist ein Federbett aufgeschüttet.

Eine Truhe steht offen und ist bis zum Rand mit grob gestampftem, würzigem Kartoffelbrei gefüllt.
Einige Schnüre sind auf einer Höhe von etwa anderthalb Schritt kreuz und quer durch den Raum gespannt.
Daran hängen ausschließlich hauchdünne Glaskugeln bis zu doppelter Faustgröße, die im kleinesten Luftzug schwingen.
Auf einem niedrigen Tisch sind verschiedene Stoffstücke in grellen Farben ausgebreitet. Darauf liegt eine riesige Gartenschere.}

Wenn sich die Gruppe etwas umgesehen hat, manifestiert sich der \textbf{Bosnickel}:
Ein Kobold in der typischen Gestalt eines Kindes mit einem unvorstellbar runzligen Greisengesicht,
gekleidet in eine schwere Ritterrüstung mit zusätzlichen Visieren an verschiedenen Körperteilen,
die gelegentlich und anlasslos auf- und zuklappen.

\zeichnung[0.45\textwidth]{Kobold.png}

\kasten{Vor Jahrhunderten hat \textbf{Xolozyhtrantantix} damit begonnen, die in den Erzadern des Bergwerks verbleibenden Edelmetalle nach und nach gegen wertlose, ja selbst gegen Unmetalle auszutauschen. Was ihn dazu bewegt hat lässt sich -- wie bei Kobolden üblich -- schlechterdings nicht in menschliche Begriffe von Ursache und Wirkung übersetzen.
	
Das \fkv sah sich nicht in der Lage, mit dem flatterhaften Wesen in den Austausch zu treten.
Die Versuche, ihn durch Gewalt zu vertreiben, waren ebenfalls zum Scheitern verurteilt. So gaben sie das Bergwerk schließlich auf.
}

\info{Spielen sie den Kobold als eine völlig andersartige Entität. Er reagiert auf Kommunikationsversuche mit Wortspielen, spontanen Gedichten -- bauen Sie dort die Hintergrundinformation aus dem nächsten Absatz ein -- und gackerndem Gelächter.}

\info{Fragen nach dem Woher, Wohin und Warum kann er nicht auf für Sterbliche verständliche Weise beantworten.\\Bei Angriffen dematerialisiert er sich.}

\enquote{Immer fader,\\ die Kupfererzader,\\
ist jetzt voll wie ein Heuschober\\ mit schönstem Zinnober.\\
Wer von den Zwergen kannte schon\\ das Antimon?\\
Dem wirklich Weisen\\ schmeckt es besser als Eisen.\\
In meinem Bergwerk ist kein Silber drin,\\ nur noch Zinn,\\
und Silber-Queck\\ und Kupfer-Neck.\\ Hab' ich wegversteckt.\\
Mein ganzer Stolz:\\ Rattengold anstatt des Golds.}

\bild[0.29\textwidth]{erz.png}

%\subsection{Sonstiges}
%Der Hauptschacht
%Daneben gibt es einige Schächte, die zum Klettern gedacht sind.
%Der Zahn der Zeit hat einige der Krampen beschädigt.
%Doppelbartschlüssel für praktischen Zugang versteckt

%Hinweise auf die Ereignisse des ersten Orkensturms.
\neuespalte
\subsection{Die zweite Ebene}
Der größte Teil der verlassenen Binge hat noch immer den Charakter eines Bergwerks.
Rostige Schienen führen in die Dunkelheit.
Immer wieder hallt das Fallen von Wassertropfen unerwartet laut durch die Stollen:
Verschiedene künstliche Wasserwege, die die Arbeit einst erleichterten, führen nach starken Regenfällen an der Oberfläche und zu Zeiten der Schneeschmelze noch immer kleine Rinnsale.

\subsubsection{Das Rattengold}
Mehrfach passiert die Gruppe \textbf{Kippe}n. Teilweise handelt es sich um taubes Gestein,
teilweise um Erze der für das \fkv wertlosen Materialien, die der Bosnickel benannt hat.

\vorlesen{
Als ihr an einer weiteren Kippe vorbeikommt, glitzern die Erzbrocken im Licht eurer Fackeln verheißungsvoll. Das Gestein scheint von reichlich Gold durchzogen zu sein.
Ein etwa faustgroßer Brocken direkt zu euren Füßen wirkt, als bestünde er hauptsächlich aus Gold.
Doch als ihr versucht, ihn anzuheben, seid ihr von seinem Gewicht überrascht.
Selbst die Stärkste von euch braucht zwei Hände, um den Brocken überhaupt vom Boden zu bekommen.
Weder Blei noch reines Gold in dieser Menge dürften nur annähernd so schwer sein.}
\spaltenende

\probenkasten[bild=alchemie,
farbe=gruen,
gruppenprobe=gut,
zusammenarbeit=ja,
pw=16,
erfolg={},
misserfolg={}
]
{Analyse, Schmieden \emph{oder} Götter und Kulte}{
Es muss sich um \textbf{Rattengold} handeln, eines der sogenannten 13 Unmetalle.\\
	\textbf{20:} Rattengold wird dem \textbf{Namenlosen} zugeordnet. Es heißt, dass Gefolgsleute dieses Gottes nicht mit dem hohen Gewicht zu kämpfen haben.\\
	\textbf{24:} Wegen seiner Affinität zum Namenlosen ist Besitz und Verarbeitung in zwölfgöttertreuen Landen verboten -- soweit überhaupt bekannt.\\
	\textbf{28:} Rattengold lässt sich für einige \textbf{Alchemika} verwenden, beispielsweise als potentes Substitut für Gold und Bernstein im \textbf{Bannstaub} \emph{(Ilaris:64)} oder für Zinnober im \textbf{Purpurblitz} (\emph{Ilaris:66}).\\
	\textbf{32:} Es heißt, dass der \textbf{Kaisertöter} -- eine mythische Klinge, mit der der Kult des Namenlosen Herrscherhäuser stürzt -- aus Rattengold geschmiedet wird.}
\spaltenanfang


\subsubsection{Die Wühlschrate}
Eine Sippe \textbf{Wühlschrate} hat sich aus der Tiefe auf die Höhe der zwergischen Tunnel vorgearbeitet.
Insbesondere einige junge Schrate haben diese dabei auch beschädigt.
Unter ihnen ist niemand über einen Schritt groß -- die Löcher sind so klein, dass Spielfiguren ihnen nicht folgen können.

\textbf{Wie ein Sembelquast}\\
Ab der zweiten Ebene gibt es in Tunneln Löcher in Boden, Decken und Wänden und Beschädigungen am Stützwerk.
\probe{wahrnehmung}{Wachsamkeit (24)}{\emph{-4 für Vorteil: Gefahreninstinkt und jede Stufe Angepasst (Dunkelheit), +8, falls keine Lichtquelle zur Verfügung steht}\\\mbox{}\\um der Gefahr zu entgehen.}
\probe{gewandtheit}{Gewandtheit (I, 12)}{um anschließend den Weg fortzusetzen.}
Schlägt eine der beiden Proben fehl, kommt es zu Einstürzen oder Einbrüchen in den löchrigen Boden mit jeweils 2W6+4 SP.
\bigskip

\textbf{Angriffe aus dem Dunkeln}\\
Die Schrate empfinden die Gruppe als Eindringlinge.
Insbesondere die Lichtquellen flößen ihnen Angst ein.
Gleichzeitig vermuten sie, dass ohne das Licht der Aufenthalt für die Gruppe sehr viel ungemütlicher wird.
Sie sind intelligent genug, Lichtquellen zu identifizieren und ein Bündel Fackeln aus einer Lagerstelle zu stehlen oder eine zur Untersuchung abgestellte Lampe zu zerstören.

Durch ihre Ortskenntnis und die von ihnen gegrabenen Gänge können die Wühlschrate in unbeachteten Momenten immer wieder zuschlagen.

\info{Spielen Sie die Schrate als Jugendbande, die ihr Zuhause zu verteidigen versucht:
Ihre Versuche, die Gruppe zu behindern, sind eher Streiche.} Beispielsweise schieben sie im Dunkeln eine Lore an, so dass sie quietschend im Lichtkreis der Fackeln zu stehen kommt.
Nutzt die Gruppe trotz dieser unheimlichen Situation die Gelegenheit zu einer Fahrt, nimmt die Lore auf der abschüssigen Strecke beachtliche Geschwindigkeit auf.
Das von den Schraten durchbissene Gleis sorgt schließlich für ein schepperndes Ende und 2W6+4~SP.



\zeichnung{Wuehlschrat.png}

Bei Angriffen mit Waffengewalt oder Licht ziehen sie sich sofort zurück. Je nach Erfolg werden sie für eine Weile eingeschüchtert.
Zu regelrechten Kämpfen dürfte es nur kommen, wenn die Gruppe beschließt, eine Falle zu stellen.

Neben der \textbf{Bärenfalle} (\emph{Ilaris:63}) sollten Sie auch kreativere Lösungen ermöglichen und belohnen.
Beachten Sie aber, dass eine solche Falle nur bewacht funktionieren kann:
Selbst ein Käfig aus Stein oder Metall stellt für das Gebiss eines Wühlschrats kein Hindernis dar und wäre in ein bis zwei Initiativephasen zerkaut.


\kreatur{Junger Wühlschrat}{scheuer Schrecken zwergischer Baukunst}{gfx/kreaturen/humanoid}{
	\kreaturWsMrGsIni{8/12}{10}{2}{2}
	Lichtscheu% Vorteile
	\linie
	\kreaturwaffe {Hieb}{0}{5} {7}   {2W+2}  {Umklammern (-4, 16)}
	\kreaturwaffe {Biss} {0} {0}{12}{4W6+6 }{Rüstungsbrechend}
	\linie
	\kreaturinfo{Attribute}{GE 6, KK 6, KL 0, KO 16, MU 4}
	\kreaturinfo{Fertigkeiten}{Klettern 14, Wachsamkeit 8, Pirschen 6, Zähigkeit 6}
}



\subsection{Das Rätsel}
\vorlesen{Als ihr die nächste Höhle betretet, entdeckt ihr an deren anderen Ende auf den ersten Blick das Ziel eurer Suche.
Von dem dreiflügligen Portal trennt euch eine breite Schlucht.
Zwar wölbt sich die mit vereinzelten Stalaktiten behangene Decke lediglich einige Schritt über euch. Jedoch geht es mindestes fünfundzwanzig Schritt in die Tiefe.
Dort wälzt sich träge das graue Magma; immer wieder platzt rot glühend eine Blase und schickt einige leuchtende orangenen Tropfen in die Höhe. Deswegen herrscht hier \emph{Mondlicht} (Ilaris:46).
Die wabernde Hitze (\emph{Kh\^omglut, Ilaris:35}) treibt euch den Schweiß aus den Poren.


Der einzige Weg auf die andere Seite ist ein schmaler, gekrümmter Rücken, anscheinend aus reinem Obsidian.
Die obere Kante verjüngt sich von etwa zwei Schritt zur Mitte hin auf einen knappen Schritt.
Ihre Oberfläche wurde abgeschliffen und poliert. Auf diesem Weg könnt ihr auf die andere Seite gelangen.
Über die letzten drei Schritt führt eine Brücke, deren Aufhängung euch verrät, dass sie nach unten weggeklappt werden kann.
}

Was von fern wie ein dreiflügliges Tor gewirkt hat, erweist sich aus der Nähe als schmales Portal.
Rechts und links sind einige \textbf{Rogolan}-Runen-Bänder zu erkennen, die jeweils eine Rune wiederholen.
Rechts unten endet ein Band in einem Amboss-Symbol.
Der Öffnungsmechanismus befindet sich in der Mitte: Drei drehbare Räder sind so in das Portal eingelassen, dass sie jeweils eine Rogolan-Rune zeigen. Sie lassen sich nach Art eines Zahlenschlosses drehen, wobei das gesamte Rogolan-Alphabet zur Auswahl steht.
Darüber ist ein reich verzierter, aber glatter und spitz zulaufender Hebel.

\info{
Geben Sie Ihrer Gruppe nun das Rätsel auf den folgenden Seiten (Lizenzhinweise auf S.\pageref{lizenz}).
}

\info{Ich empfehle, es auf zwei verschiedenen Seite auszudrucken, auszuschneiden und passgenau aufeinander zu kleben -- doppelseitiger Ausdruck gibt leider ein verschobenes Ergebnis. Legen Sie das resultierende Blatt möglichst flach auf den Tisch. Die Entdeckung der Rückseite soll schon den ersten Aha-Effekt darstellen:}

\probe{wahrnehmung}{Sinnenschärfe (16)}{\emph{Gruppenprobe, Zusammenarbeit möglich.}\\Als ihr den Stein betastet entdeckt ihr, dass die Runenbänder rechts und links jeweils in vier etwa gleich hohe Paneele geteilt sind. Sie lassen sich nach vorne klappen. Auch ihre Rückseiten sind mit Runen gezeichnet.}

\info{Schneiden Sie nun die waagrechten Linien bis zu den senkrechten ein und knicken Sie die Elemente um.\\
Je nach Geschmack Ihrer Gruppe können Sie das Rätseln den Mitspielenden überlassen oder mit einer Probe die Figuren auf entscheidende Tipps kommen lassen:
}
\spaltenende

\probenkasten[bild=klugheit,
farbe=braun,
gruppenprobe=gut,
zusammenarbeit=ja,
pw=12,
erfolg={Die Anordnung der Paneele erinnert an die Runen hinter dem \textbf{Angrosch-Altar} auf der ersten Ebene.\\
	\textbf{16:} Die drei Sorten Runen entsprechen den Buchstaben des Namens Angrosch und geben damit eine Reihenfolge (1.\,NG 2.\,R 3.\,SCH)vor.\\
	\textbf{20:} Ihr könnt euch an die einzelnen Kombinationen hinter dem Altar erinnern.\\
	\textbf{24:} Eine ununterbrochene Verbindung zwischen zwei Ambossen erfordert jeweils eine bestimmte Kombination. Diese entspricht einer der Platten am Altar und bezeichnet somit einen Buchstaben.\\
	\textbf{28:} Die richtige Reihenfolge wird durch den Namen Angroschs vorgegeben. Es ergeben sich: NG-Band: D, R-Band: G, SCH-Band: M.},
misserfolg={Wird der Hebel gezogen, während eine falsche Kombination eingestellt ist, klappt die Brücke weg und die Person am Hebel stürzt unweigerlich in die Tiefe.
	Dies könnte leicht den Tod im Magma bedeuten, so dass sich glücklich schätzen muss, wen scharfkantige Obsidianfelsen aufhalten.
	4W6+4 \emph{Sturz}schaden verursachen diese dennoch; nach einer \emph{Akrobatik-Probe} (16) die Hälfte (\emph{Ilaris:34}).
	Ist die stürzende Figur mit einem Seil gesichert, würfelt die haltende Figur
	\probe{Koerperkraft}{Körperkraft (24)}{%
		\emph{Zusammenarbeit mit 1 Person möglich}\\
		Bei Erfolg bleibt der Sturz ohne schmerzhafte Folgen.
		Bei Misserfolg rutscht auch die haltende Person auf dem glatten Grat aus und beide stürzen.}
	Das Seil erleidet an den scharfen Kanten in jedem Fall 2W6~SP (\emph{Ilaris:32}) bei jedem Sturz.},
]
{Klugheit}{\mbox}


\newpage
\handout
\fbox{\includegraphics[width=0.95\textwidth]{tor-v.png}}
\newpage
\fbox{\includegraphics[width=0.95\textwidth]{tor-r.png}}
\handoutende
\spaltenanfang 
Des Rätsels Lösung ist somit \enquote{\textbf{Dorgrim}} -- Hochkönig und Träger des \textbf{Szepters der Stärke} während des \textbf{ersten Orkensturm}s.

Wird diese Kombination eingestellt und der Hebel gezogen, versinkt der mittlere Teil knirschend im Boden und es öffnet sich für einige Minuten ein schmaler Eingang.

Die Brücke steigt allmählich wieder auf ihre ursprüngliche Position, sobald der Hebel losgelassen wurde.
\subsubsection{Die Gruft}
\vorlesen{
	Auf der Innenseite der Tür gibt es einen zweiten Hebel, mit der sich der Ausgang ebenfalls öffnen lässt.
	
In der Mitte des Raums erinnert eine drei Schritt hohe Statue -- zugleich die tragende Säule -- an \textbf{Dorgrim \enquote{Orkenbrandt} Sohn des Dargalosch}.
Das steinerne Szepter in ihrer Hand hat eine Aussparung, in die das eigentliche Szepter hineinpasst. Diese ist jedoch leer.
\\
An den Wänden erinnern Platten an große Angehörige des \fkv s und ihre wichtigsten Taten.
In den Nischen darunter liegen jedoch nur Werkzeug und Waffen. Die Körper wurden dem Schöpfer \textbf{Angrosch} zurückgegeben und in den Feuerschlot geworfen.

Dieser \textbf{Schlot} kommt in Sicht, als ihr die Statue umrundet. Doch das feurige Glosen aus der Tiefe wird durch eine weitgehend transparente, wabernde Masse verzerrt.}

Eine \link{https://ilaris-online.de/app/kreatur/175}{Riesenamöbe} (\emph{Ilaris:105}) muss durch ein von den Wühlschraten gegrabenes Loch ihren Weg in die Gruft gefunden haben.
Zu den Dingen, die sie sich hier einverleibt hat, gehört ausgerechnet das Szepter!

\info{Die angegebenen Kampfwerte setzen voraus, dass die Gruppe durch die vorhergehenden Strapazen und Fallen schon einige \textit{Einschränkungen} aufweisen. Für eine ausgeruhte Gruppe ist die Amöbe kein ernst zu nehmender Gegner. Setzen Sie in diesem Fall die Werte für ein besonders gefährliches Exemplar ruhig ein wenig hinauf.}

\kreaturriesenamoebe

Nach dem Kampf kann die Gruppe dieses bergen. Es besteht hauptsächlich aus \textbf{Asthenil}, mit Applikationen aus \textbf{Toschkril} und \textbf{Zwergengold}.
Ein großer \textbf{Serpentin} ist an zentraler Stelle eingebettet.
\\
Auch einige Waffen und Werkzeuge finden sich in den amorphen Überresten der Amöbe. Diese sind natürlich von hervorragender Qualität (3xHQ und mehr, \emph{gutes Werkzeug} (\emph{Ilaris:61})), waren hier aber eigentlich zur ewigen Ruhe gebettet. Etwas mitzunehmen könnten Angrosch und auch Boron als Frevel auffassen \dots

\subsubsection{Immer diese Schrate!}
Beim Austritt aus der Gruft stellt sich heraus, dass die Wühlschrate während des Kampfes den Mechanismus der Brücke zerbissen haben. Sie bleibt dauerhaft abgesenkt.
\spaltenende
\probenkasten[bild=athletik,farbe=rot, gruppenprobe=nein, zusammenarbeit=nein, pw=24,
erfolg={Nach dem ersten Erfolg kann ein Seil herübergeworfen werden. Weitere Proben entfallen.},
misserfolg={Ein Fehlschlag führt bäuchlings auf die harte Oberfläche: W6+4~SP.},
]
{Akrobatik}{Der Sprung auf den Grat ist leicht. Schwer ist, punktgenau auf der schmalen und glatten Oberfläche zu landen. }
\spaltenanfang

\subsection{Weitere Ebenen}
Auf der Suche nach Abstiegen nehmen die Wühlschrat-Schäden zu. Auch muss hier und da an einer Schutthalde ein Durchschlupf gegraben werden.

Mit etwas Mühe finden sich aber Wege in die Tiefe, etwa zum großen Wasserspeicher, den \textbf{Felswandler} (\emph{KK:113}) bewachen, oder einem zugewachsenen Hinterausgang, wo vor Jahrhunderten Abraum entsorgt wurde.

Weitere Geheimnisse mögen sich hier verbergen, doch sind sie nicht Teil dieses Abenteuers.

\spaltenende

\neueseite

\section{Dritter\,Akt: Das Szepter}
\spaltenanfang
\subsection{Der Rückweg}
\label{zuruck}
\vorlesen{Als ihr das \textbf{Rattengold} passiert, fällt euch auf, dass der Brocken von der Mitte des Gangs verschwunden ist. Ihr seht euch um und findet stattdessen einen handtellergroßen goldenen \textbf{Knopf} von zwergischer Machart.}

\kasten{\textbf{Bonderik} ist ein Schurke, aber kein Anhänger des Namenlosen. Der Brocken ist auch für ihn sehr schwer und er hat sich beim Einpacken den Knopf abgerissen.}

Der natürliche Anlaufpunkt nach dem Fund des Szepters ist wieder Inradon \textbf{Xermosch}.
Für den nächsten Feuertag wird das Ritual angesetzt, das Szepter neu zu schmieden.
\subsection{Die Zeremonie}
\info{Stimmen Sie mit der Gruppe ab, von wo diese das Ritual miterleben wollen:
	Angroschgläubigen Figuren können daran mitzuwirken.
	Andere können ehrenhalber einen der begehrten Plätze im Tempel einnehmen. Wer sich irgendwann des Vertrauens des \fkvs als unwürdig erwiesen hat, wird wohl in der Halle ganz hinten stehen müssen -- ein interessanter Ausgangspunkt für die \emph{Verfolgungsjagd}.}
\vorlesen{%
Im Tempel ist jeder Platz besetzt. Auch in der großen Versammlungshalle sind hunderte, wenn nicht tausende Zwerginnen und Zwerge in ihren schönsten Kettenhemden zusammengekommen. Dicht gedrängt, murmeln sie seit Stunden gemeinsam Gebete in monotonem Singsang. Schließlich ist der große Moment da: Auf ein Zeichen von Inradon \textbf{Xermosch}, der im Portal des Tempels steht, werden Pauken und Ambosse geschlagen und unter einzelnen, tiefen Posaunenstößen wird das Szepter von einer Prozession in den Tempel getragen.

Dort ist ein Muffelofen aufgebaut worden, in dem die Metalle geläutert werden sollen.
Zahlreiche Beteiligte gehen dem Hohepriester zur Hand, darunter alle \textbf{Angrosch}-Geweihten des Finsterkamms und die meisten Mitglieder des \textbf{Tiefen Rats}. In jahrtausendealten steinernen Schalen fangen sie die geschmolzenen Metalle.
Der Inradon, den Serpentin in Händen, leitet sie zum großen Altar-Amboss. Danach greift er zum Hammer: \enquote{Mit diesen Schlägen werden wir unsere Zukunft in deine Hände legen, oh einziger Gott unseres Volkes.}
}

\subsubsection{Die Enttarnung des Bösewichts}

\probe{Wahrnehmung}{Wachsamkeit (16)}{\emph{keine Zusammenarbeit möglich.} Einer mitschmiedenden Figur fällt auf, dass das geschmolzene Zwergengold in einer der Schalen in Bewegung gerät.}

\kasten{Wer weiter entfernt steht oder bewusst versucht, den Knopf zuzuordnen, bemerkt mit derselben Probenschwierigkeit, dass die Knöpfe an Bonderiks Robe dem gefundenen gleichen.}

Genaueres Hinsehen zeigt, dass an Bonderiks Ärmel ein Knopf fehlt. Sobald sich ein Gruppenmitglied bemerkbar macht, erfasst Unruhe die Mitbetenden und setzt sich in der Menge fort, auch wenn die Ursache von den Wenigsten verstanden wird. \textbf{Xermosch} unterbricht das Ritual, um die Details zu erfragen.
\textbf{Bonderik} gerät nicht aus der Ruhe, tritt aber den Rückzug an.

Nun kann eine \emph{Verfolgungsjagd} (\emph{Ilaris:57}, DG\,3) beginnen:\\
In der dicht stehenden Menge ist die Grund-GS 1. \emph{Laufen} wäre \emph{langsam}.
Bonderik drängelt sich also durch die Menge und probt auf \emph{Körperkraft}, PW\,12, um \emph{schnell} voranzukommen. Er versucht, in die Hallen seiner Sippe zu gelangen und den verräterischen Rock loszuwerden.

Wer ihn verfolgt, kann ebenfalls \emph{KK} einsetzen, oder sich mit \emph{Akrobatik} über die Schultern (darüber einigermaßen empörter) Zwerg*innen bewegen. Die Grund-GS beträgt dann 4. Diesmal ist auch möglich, stattdessen die Menge auseinander zu treiben und den EW einer Probe auf \emph{Anführen} oder \emph{Einschüchtern} der Geschwindigkeit zuzuschlagen.\\
Wird die Verfolgungsjagd gewonnen, ist der Bösewicht gestellt und kann beschuldigt werden.
\textbf{Bonderik} streitet alles ab. Die Aussagen einiger Fremder und das Vorzeigen des Knopfs reichen als Beweis nur so weit, dass er sicherheitshalber vom Ritual ausgeschlossen wird. Anschließend werden die Metalle erneut getrennt und das Szepter ohne Rattengold geschmiedet. Nach einem letzten Gebet kreiselt es auf dem Altar, bis es auf \textbf{Garbalon} zeigt.\\
Nach dessen Krönung wird \textbf{Bonderik} unzufriedene Zwerge um sich scharen und in verlassenen Bauten im Süden des Finsterkamms ein neues Volk ausrufen.
\subsubsection{Und wenn nicht?}
Wenn die Gruppe die Manipulation nicht entdeckt, dann erleben sie eben ein einzigartiges Ritual mit, an dessen Ende \textbf{Bonderik} zum Hochkönig gewählt wird.
Entkommt er bei der Verfolgungsjagd, so fehlt ein Stück der Indizienkette und es ist die Gruppe, die wegen der Störung des Rituals von der restlichen Zeremonie ausgeschlossen wird.
Haben sie zuvor das Vertrauen des Inradon gewonnen, so wird dieser dennoch sicherheitshalber von vorn beginnen.
\subsection{Abgesang}
Für die Krönung wird auch \textbf{Gerambolosch} wieder hereingetragen.
Als die Krone das Haupt des neuen Hochkönigs berührt, erwacht der Bergkönig. (\enquote{Ich kehre aus dem Labyrinth meiner Träume zurück.}) Je nachdem segnet er seinen Sohn oder prophezeit dem Verräter Unglück.
\info{Eine hervorragende Gelegenheit für Sie, durch den Mund des greisen Königs eine Prophezeiung hinzuzufügen, die ihre Gruppe ins nächste Abenteuer führt.}
Dann schließt er die Augen für immer.

\spaltenende
\section{Anhang}
\spaltenanfang




\subsection{Finsterkoppen als Abenteuer\-schauplatz}

Das \fkv siedelt seit 3.000 Jahren im \textbf{Finsterkamm}; dies dürften in den meisten Familien rund 15~Generationen sein.

\subsubsection{Großling, sei auf der Hut}
Menschen sind in der Stadt \textbf{Finsterkoppen} äußerst selten zu Gast. Zwar unterhält das \fkv regen Handel über \textbf{Lowangen} und \textbf{Gashok}.
Dieser wird jedoch hauptsächlich über das deutlich leichter erreichbare \textbf{Hilltorp} abgewickelt.
Und auch der Weg über den \textbf{N\^orrnstieg} nach \textbf{Nordhag} fällt Kaufleuten aus Finsterkoppen deutlich leicht als umgekehrt.

Somit gibt es in Finsterkoppen schlechterdings kein Gebäude, das auf menschliche Maße ausgerichtet und keine Höhle, die für Menschen auskömmlich ausgeleuchtet wäre.

Selbst in prunkvollen Räumen wie der Versammlungshalle oder dem Angroschtempel herrscht \emph{Dämmerung} (\emph{Ilaris:38}).
Überall kann eine herabhängende Leuchte, ein Durchgang oder ein Sitzmöbel zum Hindernis für jemanden von mehr als sieben Spann Körpergröße werden.

\subsubsection{Das große Tor}
Es ist durchaus möglich, aus einigen der oberirdisch zugänglichen Gebäude in die Siedlung im Berginneren zu gelangen.
Diese Wege sind jedoch nicht öffentlich zugänglich, sondern führen in der Regel über Privatgemächer und Keller durch die Wohnhöhlen der ansässigen Sippen. 
(Außerdem haben diese Vorsorge getroffen, denn der Drache schläft nicht! Im Falle eines Angriffs lassen sich alle diese Zugänge mit wenigen Handgriffen zerstören.)

Der vorgesehene Weg führt durch das \textbf{große Tor}.
Dieses Wahrzeichen Finsterkoppens ist ein beeindruckendes Zeugnis zwergischer Baukunst:
Das ungeschulte Auge vermag nicht zu sagen, welcher Teil gewachsen, welcher gemeißelt, welcher gegossen und welcher gemauert worden sein mag.
Ein Gebirgsbach wurde so geleitet, dass die letzten Schritt zum Tor über eine Brücke führen, unter der schäumend das eiskalte Wasser tost.
Zwei Statuen von Heroen der Vorzeit in doppelter Lebensgröße halten als steinerne Wächter ewige Wacht.
Zwei lebendige Wächter, die deren Nachfahren sein könnten, warten vor dem äußersten der vier Torbögen.
Jeder dieser Bögen ist etwas kleiner als der nächste und  jeder enthält eine Vorrichtung zur Verteidigung gegen Drachen oder andere Angreifende:
\begin{enumerate}\setlength\itemsep {0em}
	\item Fallgatter,
	\item Schießscharten und Pechnasen sowie
	\item steinerne Torflügel, die auf Schienen geführt und mit einem System von Gegengewichten gesteuert, von einer einzigen Hand ins Schloss geführt werden, so dass kaum eine Fuge bleibt;
\end{enumerate}

An das Tor schließt sich ein etwa zwanzig Schritt langer Gang von acht Spann Höhe an.
Dieser führt direkt in die \textbf{große Versammlungshalle}.
Seine einzige Funktion ist, im Verteidigungsfall eine Engstelle zu bilden:
Sollten angreifende Drachen und Würmer das Tor tatsächlich überwinden, so lassen sie sich hier bekämpfen.

Historisch waren sowohl dieser Gang als auch die anschließende Halle und der \textbf{Angrosch}-Tempel in ihrer Verlängerung einst Teil des zentralen Einfuhr-Schachtes der ersten Mine von Finsterkoppen.
Der Gang wurde als Wehranlage verengt, die Halle zu Repräsentationszwecken vergrößert und der Tempel ausgebaut. 
Die Verlängerung dieses Schachtes in die Tiefe wurde durch eine steilere Treppe ersetzt und sein ursprünglicher Verlauf zu verschiedenen Zwecken umgebaut. Diese Höhlen werden im \emph{zweiten Akt} erkundet.

\subsubsection{Handwerk}
Wie bei allen zwergischen Völkern wird das Handwerk in hohen Ehren gehalten und die Bergleute, Mechaniker*innen, vor allem aber die Schmiedewerkstätten bringen Ergebnisse hervor, die für die meisten Menschen unerreichbar bleiben. \textbf{Ogrim Sohn des Olgosch}, \textbf{Arombolosch Eisenarm} oder \textbf{Xagula, Tochter der Xebrima} wären unter Umständen auch bereit, einem Menschen eine Waffe zu verkaufen.
Sie als Lehrmeister*innen zu gewinnen, setzt dagegen für nichtzwergische Figuren einen großen Verdienst voraus (wie ihn dieses Abenteuer darstellen kann).

Unter Fachleuten berühmt ist auch \textbf{Gandrasch Sohn des Gengram}. Der geniale, aber leicht abzulenkende Tüftler ist dafür zuständig, die Geschütze und andere mechanische Verteidigungsanlagen für das \fkv zu warten.
In seiner wenigen freien Zeit bastelt er am liebsten an Armbrüsten und hat schon manch eine Verbesserung zu Stande gebracht -- ebenso wie eigenwillige Einzelstücke.


\subsubsection{Truppen}

Das \fkv ist immer von Feinden umgeben gewesen und es wird allseits erwartet, dass alle Erwachsenen im Fall des Falles eine Axt nicht nur zu schmieden, sondern auch zu führen verstehen.

Daneben hat das Bergkönigreich einige erfahrende \textbf{Kämpfer*innen}, die regelmäßig Hatz auf \textbf{Tatzelwürmer} und \textbf{Horndrachen} oder Orksippen planen,
die für die Siedlungen oder Siedlungsvorhaben der Zwerge zur Gefahr werden.
Die Wachdienste in Finsterkoppen selbst -- einschließlich der Wache am großen Tor -- sind jedoch so auf die Sippen verteilt,
so dass alle mal für einen Mond dran sind.
Angesichts der relativ friedlichen letzten Jahrhunderte sind diese Wachen vor allem durch ihre ausgezeichnete Ausrüstung gefährlich -- weniger, weil sie nach einer Gelegenheit suchen, diese einzusetzen.

\subsubsection{Gasthäuser}
In Finsterkoppen gibt es drei \textbf{Tavernen}: 
\begin{enumerate}
\item \textbf{\enquote{Bei Schwarzbart}} -- hier verkehren auch Jugendliche vor der Feuertaufe --,
\item \textbf{\enquote{Rote Erde}}, wo \textbf{Dragoran Sohn des Denderan} vor allem Bergleute begrüßt und \item \textbf{\enquote{Hammer und Amboss}} von Wirt \textbf{Vothan Dengeler} -- wo mehr Handwerker*innen anzutreffen sind.
\end{enumerate}

\subsubsection{Heiler}
Wer an Krankheiten leidet oder zahlreiche Wunden zählt, kann den \textbf{Heiler Thoram Sohn der Cadrima} aufsuchen.
Allerdings ist auch dieser auf zwergische Anatomie spezialisiert und wird den ein oder anderen Fluch, wie viel Verbandsmaterial auf weiches Großling-Fleisch verwendet werden muss, nicht unterdrücken können.

Wer jeglichem Aberglauben abhold ist, kann sein Glück auch bei \textbf{Turgol} suchen, dem Berater des Bergkönigs, der sich in letzter Zeit besonders häufig in der Stadt aufhält.
Von der Bevölkerung wird dieser Name allerdings meist nur gemurmelt und von einem Besuch abgeraten.
Noch weniger als seine Hilfe in Anspruch nehmen würden sie jedoch ein wirklich schlechtes Wort sagen -- so sehr ängstigt sie allein der Gedanke an den (eigentlich sehr gutwilligen) Geoden.

\subsubsection{Sonstige Besonderheiten}
An den Hängen der umliegenden Berge grasen \textbf{Ponys}, die unter Tage Maschinen für den Bergbau betreiben. Die Monde von Hesinde bis Tsa verbringen sie in Stallhöhlen.


\subsubsection{Weitere wichtige Bewohner}

Die Beschreibung der Mitglieder des \textbf{Tiefen Rat}s finden Sie im \emph{ersten Akt} auf S.\,\pageref{rat}. 

\bigskip

\textbf{Muragolosch}

Der Schüler \textbf{Turgol}s schlägt im Wesen nicht nach dem gutmütigen Sumudiener, sondern hat sich dem Element Eis zugewandt.
Seit Jahrzehnten wächst seine Verachtung für andere Völker.
Er ist einer der wenigen Vertrauten \textbf{Bonderik}s und vergiftet in seinem Auftrag Bergkönig \textbf{Gerambolosch} in seinen (beklagenswerten, aber beschwiegenen) Zustand.

Falls die Gruppe auf diese Spur stößt, kann \textbf{Muragolosch} eine größere Rolle spielen. Im unwahrscheinlichen Fall eines Kampfes gegen \textbf{Bonderik} kann er als Joker eingreifen und diesen mit Zaubern wie \biglittlecap{Weg durch Sumus Leib} in Sicherheit bringen.

\bigskip

\textbf{Arglescha Tochter der Angrarda}

Die Tochter \textbf{Garbalon}s ist erst ein halbes Jahrhundert alt, gilt aber dennoch schon jetzt als eine der besten Kämpferinnen im \fkv.
Jüngst zeigt sie Interesse, auch ihre Fähigkeiten als Anführerin auszubauen -- ein Vorhaben, das durch die Wahl ihres Vaters zum Bergkönig noch plausibler wird.

Arglescha eignet sich hervorragend als romantisches Interesse für einen Zwergenkavalier als Spielerfigur.

\spaltenende
\begin{center}
\zeichnung[0.95\textwidth]{zwerg.JPG}
\end{center}

\newpage
%\subsection{Beispiel-Charaktere}
%\paragraph{Graboxa, Zwergische Angrosch-Geweihte, Hüterin der Wacht}
%\paragraph{Dascha, Gjalsker Tierkriegerin}
%\paragraph{Jasinthe, Garether Gelehrte}
%\paragraph{Odir Sensendengler, Abenteurer und Pferdedieb}
\subsection{Szenen/Eigenheiten-Tabelle}
Hier eine Übersichts-Tabelle, mit welchen Eigenheiten Sie den Figuren in welcher Szene \emph{das Leben schwer machen} \emph{(Ilaris:15)} können, um Schicksalspunkte zu verteilen:

\begin{tabularx}{0.98\linewidth}{l|XXX}
	&\tkopf{Der Tiefe Rat} & \tkopf{Die Finsterkopp-Binge}&\tkopf{Das Szepter}\\
	\hline
	
	\textbf{Graboxa}&\textbf{Alter bringt Weisheit}\newline Graboxa könnte sich einer im Rat geäußerten Meinung einfach deswegen anschließen, weil sie von einer deutlich älteren Person kommt. Bieten Sie einen Schicksalspunkt, wenn daraus Spannungen mit der Gruppe entstehen.
	&\textbf{Für Angrosch!}\newline Graboxa empfindet unmittelbar den Frevel, die Artefakte nicht wieder in die zugehörigen Nischen zu legen. Gönnen Sie ihr einen Schicksalspunkt, wenn sie die anderen davon überzeugt, auf diese handfeste Belohnung zu verzichten.
	& \textbf{Drachendiener überall}\newline Warum hat der Hohepriester beim Gebet die Einzigkeit des Gottes so betont? Und warum hat er ihn nicht beim Namen genannt?? Eine vorübergehende Verwirrung kurz vor Schluss darüber, wer hier (eigentlich) alles Übles im Schilde führt, gibt einen Schicksalspunkt.\\
	
	\textbf{Dascha}&\textbf{Gewalt ist immer eine Lösung}\newline Wenn Dascha vorschlägt, den greisen Bergkönig einfach zu erschlagen, ist das schon einen Schicksalspunkt wert -- selbst dann, wenn es der Gruppe gelingt, sich auf einen Übersetzungsfehler rauszureden. &\textbf{Singender Stahl}\newline Für Dascha ist es ein Unding, die hervorragenden Äxte einfach allesamt in der Gruft liegen zu lassen. Gönnen Sie ihr einen Schicksalspunkt, wenn sie sich darüber mit Graboxa überwirft.&\textbf{Trolltochter}\linebreak Bei Daschas Größe ist nicht die Frage, ob sie sich den Kopf stößt, sondern wie oft \dots\ Während der Verfolgungsjagd kann Dascha für einen Schicksalspunkt auf einen Würfelwurf verzichten und den Vergleich verlieren.\\
	
	\textbf{Jasinthe}&\textbf{Wisst ihr eigentlich \dots ?}\newline Wenn den Rat eine Sache nicht interessiert, dann wie die Eslamiden eine ähnliche Situation gelöst haben. Für eine peinlich-weitschweifige Erklärung könnte es einen Schicksalspunkt geben.&\textbf{Wunder der Technik}\newline Wenn Jasinthe begeistert darauf eingeht, wie gut die zwergische Technik auch nach Jahrhunderten ohne Pflege noch funktioniert und dabei als erste in die Bolzenfalle tritt, gibt es einen Schicksalspunkt.&\textbf{Chronistin}\newline Nein, der Hohepriester möchte nicht die vorige Strophe des Gebets noch einmal wiederholen, damit es mitgeschrieben werden kann. Ob wir jetzt endlich mit dem heiligen Ritual fortfahren können \dots?\\
	
	\textbf{Odir}&\textbf{Baliho? Nie dagewesen \dots}\newline Ein Lederwaren-Händler will in Odir einen Mann erkennen, der beim Pferdemarkt in Baliho am Pranger stand. Es braucht einigen Aufwand, ihn vom Gegenteil (\enquote{Menschen sehen doch alle gleich aus!}) zu überzeugen.&\textbf{Wo ist meine Hasenpfote?}\newline Dass die Wühlschrate Lichtquellen stehlen, ist das eine. Aber dass sie dabei auch den Glücksbringer haben mitgehen lassen! Die Suche nach dem Szepter muss sofort unterbrochen werden. Was ist ein Szepter gegen die Hasenpfote?!&\textbf{Gold, was glänzt}\newline Einerseits ist der goldene Knopf ein wichtiges Indiz, andererseits ist er auch einiges wert. Wenn Odir eindringlich andere -- schlechtere -- Wege der Anklage sucht, um den Knopf behalten zu können, kann das mit einem Schicksalspunkt belohnt werden.\\

\end{tabularx}


\neueseite

\handout
\subsection{Das Relief hinter dem Angrosch-Altar}
%Lösung: D G M
\label{karten}
\mbox{}

\begin{center}
	\bigskip
	\vfill
	\includegraphics[width=0.5\textwidth]{ngrsch.png}
\end{center}

\vfill

\begin{multicols}{3}
	\karte{
		\begin{multicols}{3}\mbox{ }\linebreak \begin{tabularx}{1.5cm}{|X|}
				\leer \leer \amboss \amboss
				\hline	\end{tabularx}	\neuespalte
			\begin{center} \mbox{ } \linebreak[4]
				\includegraphics[width=\columnwidth]{S.png}
			\end{center} \neuespalte \mbox{ } \begin{tabularx}{1.5cm}{|X|}
				\amboss \leer \leer \leer
				\hline \end{tabularx} \end{multicols}
	}\vfill
	
	\karte{
		\begin{multicols}{3}\mbox{ }\linebreak \begin{tabularx}{1.5cm}{|X|}
				\amboss \leer \amboss \leer
				\hline	\end{tabularx}	\neuespalte
			\begin{center} \mbox{ } \linebreak[4]
				\includegraphics[width=\columnwidth]{B.png}
			\end{center} \neuespalte \mbox{ } \begin{tabularx}{1.5cm}{|X|}
				\leer \amboss \leer \amboss
				\hline \end{tabularx} \end{multicols}
	}\bigskip
	
	\karte{
		\begin{multicols}{3}\mbox{ }\linebreak \begin{tabularx}{1.5cm}{|X|}
				\amboss \leer \leer \amboss
				\hline	\end{tabularx}	\neuespalte
			\begin{center} \mbox{ } \linebreak[4]
				\includegraphics[width=\columnwidth]{G.png}
			\end{center} \neuespalte \mbox{ } \begin{tabularx}{1.5cm}{|X|}
				\leer \amboss \amboss \leer
				\hline \end{tabularx} \end{multicols}
	}\bigskip
	
	
	\karte{
		\begin{multicols}{3}\mbox{ }\linebreak \begin{tabularx}{1.5cm}{|X|}
				\amboss  \amboss \leer \leer
				\hline	\end{tabularx}	\neuespalte
			\begin{center} \mbox{ } \linebreak[4]
				\includegraphics[width=\columnwidth]{N.png}
			\end{center} \neuespalte \mbox{ } \begin{tabularx}{1.5cm}{|X|}
				\leer  \leer \leer \leer
				\hline \end{tabularx} \end{multicols}
	}\bigskip
	
	\karte{
		\begin{multicols}{3}\mbox{ }\linebreak \begin{tabularx}{1.5cm}{|X|}
				\leer \leer \leer \amboss
				\hline	\end{tabularx}	\neuespalte
			\begin{center} \mbox{ } \linebreak[4]
				\includegraphics[width=\columnwidth]{F.png}
			\end{center} \neuespalte \mbox{ } \begin{tabularx}{1.5cm}{|X|}
				\amboss  \amboss \amboss \leer 
				\hline \end{tabularx} \end{multicols}
	}\bigskip
	
	\karte{
		\begin{multicols}{3}\mbox{ }\linebreak \begin{tabularx}{1.5cm}{|X|}
				\leer \amboss \amboss \amboss
				\hline	\end{tabularx}	\neuespalte
			\begin{center} \mbox{ } \linebreak[4]
				\includegraphics[width=\columnwidth]{J.png}
			\end{center} \neuespalte \mbox{ } \begin{tabularx}{1.5cm}{|X|}
				\amboss \leer \leer \leer
				\hline \end{tabularx} \end{multicols}
	}\bigskip

	
	\karte{
		\begin{multicols}{3}\mbox{ }\linebreak \begin{tabularx}{1.5cm}{|X|}
				\leer \leer \leer  \amboss
				\hline	\end{tabularx}	\neuespalte
			\begin{center} \mbox{ } \linebreak[4]
				\includegraphics[width=\columnwidth]{H.png}
			\end{center} \neuespalte \mbox{ } \begin{tabularx}{1.5cm}{|X|}
				\amboss \amboss \leer \leer
				\hline \end{tabularx} \end{multicols}
	}\bigskip
	
	\karte{
		\begin{multicols}{3}\mbox{ }\linebreak \begin{tabularx}{1.5cm}{|X|}
				\leer \amboss \amboss \leer
				\hline	\end{tabularx}	\neuespalte
			\begin{center} \mbox{ } \linebreak[4]
				\includegraphics[width=\columnwidth]{P.png}
			\end{center} \neuespalte \mbox{ } \begin{tabularx}{1.5cm}{|X|}
				\amboss \leer \leer \leer
				\hline \end{tabularx} \end{multicols}
	}\bigskip
	
	\karte{
		\begin{multicols}{3}\mbox{ }\linebreak \begin{tabularx}{1.5cm}{|X|}
				\leer \leer \leer \leer
				\hline	\end{tabularx}	\neuespalte
			\begin{center} \mbox{ } \linebreak[4]
				\includegraphics[width=\columnwidth]{D.png}
			\end{center} \neuespalte \mbox{ } \begin{tabularx}{1.5cm}{|X|}
				\amboss \leer \amboss \amboss
				\hline \end{tabularx} \end{multicols}
	}\bigskip
	
	
	\karte{
		\begin{multicols}{3}\mbox{ }\linebreak \begin{tabularx}{1.5cm}{|X|}
				\leer \amboss \leer \amboss
				\hline	\end{tabularx}	\neuespalte
			\begin{center} \mbox{ } \linebreak[4]
				\includegraphics[width=\columnwidth]{W.png}
			\end{center} \neuespalte \mbox{ } \begin{tabularx}{1.5cm}{|X|}
				\leer \leer \leer \leer
				\hline \end{tabularx} \end{multicols}
	}\bigskip
	
	
	\karte{
		\begin{multicols}{3}\mbox{ }\linebreak \begin{tabularx}{1.5cm}{|X|}
				\leer \amboss \amboss \amboss
				\hline	\end{tabularx}	\neuespalte
			\begin{center} \mbox{ } \linebreak[4]
				\includegraphics[width=\columnwidth]{R.png}
			\end{center} \neuespalte \mbox{ } \begin{tabularx}{1.5cm}{|X|}
				\leer \leer \leer \leer
				\hline \end{tabularx} \end{multicols}
	}\bigskip
	
	\karte{
		\begin{multicols}{3}\mbox{ }\linebreak \begin{tabularx}{1.5cm}{|X|}
				\amboss \amboss \leer \amboss
				\hline	\end{tabularx}	\neuespalte
			\begin{center} \mbox{ } \linebreak[4]
				\includegraphics[width=\columnwidth]{M.png}
			\end{center} \neuespalte \mbox{ } \begin{tabularx}{1.5cm}{|X|}
				\leer \leer \amboss \leer
				\hline \end{tabularx} \end{multicols}
	}\bigskip
	
\end{multicols}
\vfill
\footnotesize
\textbf{Lizenzhinweise:}
\label{lizenz}
\bigskip

Rogolan-Font von Thorsten Most auf Basis der Spielhilfe \enquote{Angroschs Kinder}.

Ambosssymbole: Anvil Impact von Lorc, \href{https://game-icons.net/1x1/lorc/anvil-impact.html}{game-icons.net} unter der \href{https://creativecommons.org/licenses/by/3.0/}{CC BY 3.0-Lizenz}.

Hebel-Symbol: Lever icon von Lorc, \href{https://game-icons.net/1x1/lorc/lever.html}{game-icons.net}{CC BY 3.0-Lizenz}.
\normalsize
