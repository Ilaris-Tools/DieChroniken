\newcommand{\kreaturhummeriersoldat}{\kreatur{Hummerier Soldat}{Schreckliche Ausgeburt der endlosen Tiefen, großer Gegner}{gfx/kreaturen/mythen}{\kreaturkampfwerte{6/11}{9}{2/5 (Land/Wasser)}{4}\trennlinie \kreaturvorteile{Angepasst IV (Wasser), Natürliche Rüstung, Amphibisch}\trennlinie \kreaturwaffe{Scheren}{0}{10}{10}{3W6+2}{Doppelangriff, Rüstungsbrecher, Umklammern (-4)}\kreaturwaffe{Partisane}{2}{11}{11}{3W6+7}{}\kreaturkampfvorteile{Kraftvoller Kampf II, Ausfall, Offensiver Kampfstil, Niederwerfen, Hammerschlag}\trennlinie \kreaturattribute{CH 4, FF 2, GE 4, IN 8, KK 16, KL 6, KO 16, MU 10}\kreaturfertigkeiten{Einschüchtern 12, Wachsamkeit 6, Zähigkeit 14}\kreaturinfo{Profane Vorteile}{Abgehärtet I\&II, Schnelle Heilung, Muskelprotz, Zerstörerisch II, Willensstark I}}}
%\trennlinie \kreaturinfo{Quelle}{\href{https://dsaforum.de/download/file.php?id=14259}{Krustentiere}}}


\newcommand{\kreaturhummerierelementarist}{\kreatur{Hummerier Elementarist}{Gebieter des Unwassers, 'Artillerie' der Hummerier. Großer Gegner.}{gfx/kreaturen/mythen}{\kreaturkampfwerte{5/9}{14}{2/5 (Land/Wasser)}{4}\trennlinie \kreaturvorteile{Angepasst IV (Wasser), Natürliche Rüstung, Amphibisch}\trennlinie \kreaturwaffe{Scheren}{0}{8}{8}{3W6}{Doppelangriff, Rüstungsbrecher, Umklammern (-4)}\kreaturwaffe{Spieß}{2}{10}{10}{3W6+2}{}\trennlinie \kreaturattribute{CH 16, FF 2, GE 4, IN 8, KK 14, KL 16, KO 12, MU 16}\kreaturfertigkeiten{Autorität (alle) 16, Wachsamkeit 12, Zähigkeit 12, Elementarkunde 14, Geschichten und Legenden 12}\kreaturinfo{Profane Vorteile}{Abgehärtet I, Schnelle Heilung, Willensstark I+II}\trennlinie \kreaturinfo{AsP}{52}\kreaturinfo{(Un-)Wasser}{17 (Aquasphaero, Herbeirufung des Wassers, Wasserwand, Leib der Wogen, Mahlstrom, Zorn der Elemente)}\kreaturinfo{Magische Vorteile}{Gefäß der Sterne, Verbotene Pforten, Astrale Regeneration, verbesserte  Astrale Regeneration, Bändiger der Elemente, Meister der Wünsche, Kraftlinienmagie, Effezientes Zaubern, Tradition der Hummerier III (Wasserzauber 1x MM extra, ermöglicht Zeit lassen)}}}

\newcommand{\kreaturhummerierpriesterkrustentiere}{\kreatur{Hummerier Priester}{Anführer und Veteran der Truppen der tiefen Tochter. Paktierer. Großer Gegner.}{gfx/kreaturen/mythen}{\kreaturkampfwerte{8/14}{14}{2/5 (Land/Wasser)}{4}\trennlinie \kreaturvorteile{Angepasst IV (Wasser), Natürliche Rüstung, Amphibisch, Paktierer III}\trennlinie \kreaturwaffe{Scheren}{0}{12}{12}{3W6+3}{Doppelangriff, Rüstungsbrecher, Umklammern (-4)}\kreaturwaffe{Partisane}{2}{11}{11}{3W6+8}{}\kreaturkampfvorteile{Kraftvoller Kampf III, Ausfall, Offensiver Kampfstil, Niederwerfen, Hammerschlag}\trennlinie \kreaturattribute{CH 18, FF 2, GE 4, IN 8, KK 18, KL 10, KO 18, MU 18}\kreaturfertigkeiten{Autorität (alle) 18, Wachsamkeit 10, Zähigkeit 18, Götter und Kulte 12}\kreaturinfo{Profane Vorteile}{Abgehärtet I\&II, Schnelle Heilung, Muskelprotz, Zerstörerisch I+II, Willensstark I+II, Haltet Stand!, Keine Gefangenen!}\trennlinie \kreaturinfo{GuP}{28}\kreaturinfo{Dämonische Hilfe}{15 (Dämonische Stärkung (Derekunde, KO, MU), Meister der Maritimen, Gebieter der Gezeiten, Ertränken)}\kreaturinfo{Karmale Vorteile}{Erzdämonische Tradition III, Dämonische Waffe (wie gesegnete Waffe)}}}

\newcommand{\kreaturhummerierleibgarde}{\kreatur{Hummerier Leibgarde}{Hummerischer Elitekrieger und Beschützer, großer Gegner}{gfx/kreaturen/mythen}{\kreaturkampfwerte{6/11}{14}{3}{10}\trennlinie \kreaturvorteile{Angepasst IV (Wasser), Natürliche Rüstung, Amphibisch}\trennlinie \kreaturwaffe{Schere}{0}{17}{14}{3W6}{Rüstungsbrecher, Umklammern (-4)}\kreaturwaffe{Partisane}{2}{17}{14}{3W6+5}{}\kreaturwaffe{Scherenschild}{0}{17}{14}{1W6+4}{}\kreaturkampfvorteile{Schildkampf III, Ausfall, Offensiver Kampfstil, Defensiver Kampfstil, Gegenhalten, Niederwerfen, Hammerschlag}\trennlinie \kreaturattribute{CH 8, FF 2, GE 4, IN 12, KK 16, KL 6, KO 16, MU 16}\kreaturfertigkeiten{Einschüchtern 14, Wachsamkeit 11, Zähigkeit 16}\kreaturinfo{Profane Vorteile}{Abgehärtet I\&II, Schnelle Heilung, Muskelprotz, Zerstörerisch I+II, Willensstark I+II}}}

\newcommand{\kreaturkrabbenschwarm}{\kreatur{Krabbenschwarm}{Schwarm von winzigen Krabben. Großer Gegner}{gfx/kreaturen/daimonid}{\kreaturkampfwerte{1, Koloss I}{1 (für ein einzelnes Tier)}{2}{1}\trennlinie \kreaturvorteile{Angepasst IV (Wasser), Amphibisch, Schmerzimmun II, Verwundbarkeit IV (Flächenschaden), Wundblockade (s.u.) . }\trennlinie \kreaturwaffe{Scherenschnitte}{0}{10}{10}{1W6+3}{}\kreaturkampfvorteile{Zusätzliche Attacke I, Immunität (Niederwerfen, Umreißen, u.ä.)}\trennlinie \kreaturfertigkeiten{Wachsamkeit 16, Pirschen 16}\trennlinie \kreaturinfo{Varianten}{Großer Schwarm: Koloss II, Zusätzliche Attacke III.} \kreaturinfo{Wundblockade}{Angriffe mit Einzelziel können höchstens einen Kratzer auf einmal anrichten.}}}

\newcommand{\kreaturkrabbe}{\kreatur{Hundsgroße Krabbe}{Hungrige, gierige Laurer. Kleiner Gegner.}{gfx/kreaturen/daimonid}{\kreaturkampfwerte{2/5 (dicker Panzer)}{4}{4}{1}\trennlinie \kreaturvorteile{Angepasst IV (Wasser), Amphibisch }\trennlinie \kreaturwaffe{Scheren}{0}{8}{8}{2W6}{Rüstungsbrecher, Doppelangriff}\trennlinie\kreaturfertigkeiten{Wachsamkeit 8, Pirschen 8}}}

\newcommand{\kreaturgrossekrabbe}{\kreatur{Schafsgroße Krabbe}{Hungrige, gierige Laurer. Verwachsen und mutiert.}{gfx/kreaturen/daimonid}{\kreaturkampfwerte{3/8 (verwachsener, dicker Panzer)}{4}{2}{1}\trennlinie \kreaturvorteile{Angepasst IV (Wasser), Amphibisch }\trennlinie \kreaturwaffe{Scheren}{0}{6}{6}{2W6+4}{Rüstungsbrecher, Doppelangriff}\trennlinie\kreaturfertigkeiten{Wachsamkeit 8, Pirschen 4}}}

\newcommand{\kreaturmutterkrabbe}{\kreatur{Orzugath}{Gehörnter Dämon. Großer Gegner}{gfx/kreaturen/daemon}{\kreaturkampfwerte{5/10, Koloss I}{4}{3}{1}\trennlinie \kreaturvorteile{Angepasst IV (Wasser), Regeneration I, Verseuchung (s.u.)}\trennlinie \kreaturwaffe{Große Schere}{0}{6}{14}{3W6+6}{Umklammern (-8)}\kreaturwaffe{Scharfe Schere}{0}{6}{14}{3W6+6}{Rüstungsbrecher}\kreaturkampfvorteile{Zusätzliche Attacke I}\trennlinie \kreaturattribute{GE 8, KK 28, KL 8}\kreaturfertigkeiten{Wachsamkeit 14, Pirschen 2} \trennlinie 
\kreaturinfo{Beschwörung}{Invocatio 24}\kreaturinfo{Dienste}{Gewässer versuchen +0 (7 Monate), Kampf (1 Minute) -8} %\trennlinie 
\kreaturinfo{Verseuchung}{Orzugath verflucht Krustentiere der Umgebung zu unendlichem Hunger und maßlosem Wachstum. Sie sorgt zudem für steten Nachschub, indem sie ausgiebig Laicht.}}}

\credit{Autor}{Arne Strehlow}

\credit{Illustrationen}{Bernhard Eisner}

\spaltenanfang
\absatz{Generelle Gedanken}
% \unterabsatz{Ein Rant vorweg}
\enquote{Encounterdesign hört sich nach DnD an}, werden wohl einige DSA-Spielleiter jetzt denken. \enquote{Sowas passt nicht zu DSA.}
Und mit dieser Begründung setzen sie ihren Spielern lieblose Kämpfe vor, typischerweise etwa so:
\enquote{Da sind 5 Räuber, zu jedem von euch kommt einer, und zu dir zwei. Der bei Alrik scheint der Anführer zu sein.}

Das ist nun aber eine kleinere Katastrophe für den Spielspaß, da mir zwei grundlegenden Entscheidungen für diesen Kampf bereits abgenommen wurden:
Wo laufe ich hin und auf wen haue ich drauf?

Stattdessen bleibt mir nur noch die Wahl ob ich ein Manöver ansage.
Das ist im wesentlichen eine mathematische Optimierungsaufgabe und es gibt diesbezüglich zwei Arten von Spielern:
Jene, die so etwas interessiert und solche, die es nicht interessiert.
Erstere haben die Aufgabe bald gelöst und sie wird ihnen daher langweilig. Letztere fanden sie noch nie spannend.

Ein Kampf muss meiner Meinung nach daher mehr bieten als die Manöverwahl.
Im folgenden soll es darum gehen, wie man es anders machen kann.
Hoffentlich besser, aber letztlich ist das natürlich Geschmackssache.
Zunächst ein paar generelle Tipps.

\unterabsatz{Dramatische Frage}
Damit ein Kampf spannend ist, müssen die Spieler ein Interesse an dessen Ausgang haben.
Dabei kann es hilfreich sein, eine dramatische Frage zu formulieren, die durch den Kampf beantwortet werden soll.
Zum Beispiel: \enquote{Werden alle Helden es schaffen, aus der Burg zu entkommen?},
oder: \enquote{Wird der Efferdgeweihte das Ritual der Reinigung beenden können?}

Diese Frage ist nicht dazu da, ausgesprochen zu werden, sondern ist eine Hilfestellung für die Spielleitung.
Sie klärt die Motivation von SCs und Gegnern, gibt Ideen für einen interessanten Kampfverlauf und ordnet die aktuelle Szene in die übergreifende Handlung ein.
Vielleicht wird durch sie auch klar, dass es noch andere Lösungswege als Gewalt gibt.

\unterabsatz{Zwischenziele}
Wie jede Szene im Rollenspiel sollte auch ein Kampf Entscheidungen für die Spieler bereithalten.
Oben genannt hatten wir bereits: Wo laufe ich hin, auf wen haue ich drauf und mit welchem Manöver?
Das kann uns durchaus erst einmal reichen.
Allerdings müssen diese Entscheidungen auch irgendwie spannend sein.

Das kann man erreichen, indem man Zwischenziele anbietet.
Bei der Flucht aus der Burg könnte zum Beispiel eine abseits stehende Wache mit Armbrust ein Dilemma anbieten.
Nehme ich mir die Zeit, dorthin zu laufen und die Armbrust kaputt zu schlagen, oder riskiere ich den Bolzen im Rücken?
Im besten Fall erzählt ein Kampf durch solche Entscheidungen eine eigene kleine Geschichte.

\unterabsatz{Bodenplan}
Ich empfehle für alle Kämpfe mit mehr als zwei Beteiligten, einen Bodenplan zu nutzen.
Das kann auch eine schnelle Skizze sein.
Denn wenn: \enquote{Wo laufe ich hin und auf wen haue ich drauf?} eine interessante Entscheidung sein soll, wird sonst meist viel Spielzeit verbrannt für Nachfragen der Art: \enquote{Wie weit ist der weg? ... 17 Schritt? Mist, ok, dann verlängere ich einmal Reichweite ... Warte, dann schaffe ich keine Mächtige Magie, das lohnt sich dann nicht. Dann mache ich, äh, ...}.
Hat man dagegen einen genauen Bodenplan, kann der betroffene Spieler sich schon während die anderen dran sind überlegen, ob er seinen Ignifaxius mit erhöhter Reichweite sprechen möchte, näher ran läuft, einen Feuerelementar beschwört oder sich einfach versteckt. Eine grobe Skizze leistet dies zwar nicht ganz, klärt aber zumindest ungefähr, wer wo steht und ist oft ein guter Kompromiss.

\vfill

\zeichnung{tulamid.jpg}

\neueseite

\absatz{Rollen im Kampf}
Oft ist es schwierig für die SL, neben der korrrekten regeltechnischen Umsetzung des Kampfes auch noch für alle Gegner interessante und glaubwürdige Handlungen zu improvisieren.
Als Hilfestellung kann er die Gegner in Rollen einteilen, wie zum Beispiel \textit{Anführer}, \textit{Frontsau} oder \textit{Artillerie}.
Sie geben dem Spielleiter eine Idee, wie der entsprechende Gegner sich im Kampf verhalten wird.
Ein Gegner kann dabei mehrere Rollen gleichzeitig ausfüllen. So könnte der \textit{Anführer} zugleich Armbrustschütze sein (\textit{Artillerie})
oder -- mit Plattenrüstung und Hellebarde -- \textit{Frontsau}. 

Diese Rollen sind dabei bewusst zunächst einmal unabhängig von Spielwerten, auch wenn sich bestimmte Gegner natürlich jeweils mehr oder weniger anbieten.
Aber auch ein unerfahrener Bauer kann ein Anführer sein, wenn es darum geht seine Familie zu rächen.
Wie viele und welche Rollen es genau gibt, ist Geschmacksache.
Letztlich dienen sie nur der Orientierung.
Wir geben folgend eine Beispieleinteilung.

Einen Kampf mit verschiedenen Rollen zu spicken, erzeugt nahezu automatisch interessante Entscheidungen für die Spieler.
Zu Bedenken ist allerdings auch, ob eine kunterbunte Gegnertruppe auch noch Sinn in der Spielweltlogik macht.
Die Kampfbegegnung mit einer abgerissenen Räuberbande zum Beispiel wird sicherlich spannender, wenn die Gegner in erhöhter Position einen Feuerball-formenden Magier haben.
Aber warum sollte der sich mit solchem Pack abgeben?
In solchen Fällen musst du abwägen, ob deinen Spielern eine konsistente Spielwelt, oder eine spannende(re) Kampfbegegnung wichtiger ist. 
Im besten Fall findest du eine passende Alternative oder eine Erklärung.
Vielleicht arbeitet sich statt dem Magier einer der Räuber an einem Felsbrocken ab, der wohl direkt zu Beginn des Hinterhaltes auf die SCs stürzen sollte, sich aber aktuell noch sträubt.
Oder vielleicht sind die Räuber doch nicht so abgerissen, sondern gehören zu einer Gruppe Deserteure der Dritten Dämonenschlacht, inklusive Kampfmagier.

\unterabsatz{Anführer}
Wie der Name nahelegt, koordiniert und motiviert der Anführer seine Truppen.
Durch regeltechnische Kommandos kann er zum Beispiel ihre Kampfwerte verbessern.
Noch wichtiger für den Kampfverlauf kann aber sein, dass er clevere Taktiken anleitet, wie \enquote{haltet die Tür}, oder \enquote{schießt auf den Magier!}.

Vielleicht kämpfen die Gegner sogar überhaupt nur, weil sie seine Befehle ausführen.
Sollte der Anführer fallen oder Gefangen genommen werden, fliehen sie dementsprechend womöglich.
Bei der Entscheidung, ob dies passend ist, kann dir vielleicht die dramatische Frage helfen.

Beispiele: Söldnerkommandant (\ilaris{107}). Hummerier-Priester (s.\,u.).

\unterabsatz{Artillerie}
Die Artillerie bedeckt die SCs aus größerer Distanz oder aus schwer zu erreichender Position mit Fernkampfangriffen oder Zaubern.
Besonders nahkampfstark sieht sie nicht aus, aber will ich die Zeit investieren und eventuelle Passierschläge in Kauf nehmen um dort rüber zu laufen?

Beispiele: Burgwache mit Armbrust. Elementarist (s.\,u.).



\unterabsatz{Frontsau}


Die Frontsau ist ein starker Nahkämpfer, der sowohl austeilen als auch einstecken kann.
Sie ist üblicherweise das naheliegenste Ziel für einen Angriff der Helden, wenn auch nicht unbedingt das taktisch klügste.
Ignoriert man sie, bedankt sie sich mit voller Offensive, Passierschlägen oder sie zerlegt einfach nur deinen Magier.

Beispiele: Kriegsoger (\ilaris{203}). Hummerier-Soldat (s.\,u.). 

\zeichnung{Kämpfer.jpg}

\unterabsatz{Assassine}
Der Assassine taucht bevorzugt bei einzeln stehenden, verwundbaren SCs auf.
Dazu setzt er überlegene Mobilität und/oder Heimlichkeit ein.

Beispiele: Kräftige Stallmagd, die sich spontan entscheidet, den fliehenden Gruppenmagier zu tackeln, Säbelzahntiger (\ilaris{106}).
\unterabsatz{Angsthase}
Der Angsthase hält sich zurück. Sein Hauptziel ist, heute Abend heil zu seiner Familie nach Hause zu kommen.
Er macht im Kampf genug, um nicht als Verräter dazustehen, führt explizite Befehle aus, verteidigt seine Kameraden und kämpft um sein Leben, wenn er in die Enge getrieben wird.

Beispiel: Landwehr.
\unterabsatz{Beschützer}
Der Beschützer sieht es als seine Rolle, einen oder mehrere NSC zu schützen.
Zum Beispiel den Beschwörer, der gerade einen garstigen Dämonen ruft.

Beispiel: Schildkämpfer mit Schildkampf III (Schildwall), ein Gegner mit langer und gefährlicher Waffe (Passierschläge), ein Leibmagier mit Wandzaubern (Fortifex oder Weiches Erstarre)
\unterabsatz{Zeitdrücker}
Der Zeitdrücker macht Zeitdruck, indem er irgendeiner Tätigkeit nachgeht, die besser nicht vollendet werden sollte.

Beispiel: Wache, die zum Fallgitter-Hebel läuft. Magier, der sich auf einen Zauber konzentriert.

\absatz{Regeltechnisches}
Oben haben wir einige Rollen beschrieben, die die SL den Gegnern zuweisen kann, um den Kampf interessanter zu gestalten.
Dabei ging es vor allem um deren Verhalten. Ein anderer Aspekt ist die regeltechnische Ausgestaltung.
Auch hier können unterschiedliche Gegner für interessante Entscheidungen sorgen.
Greife ich lieber den Barbaren mit der großen Axt an, auch wenn er ein Kettenhemd trägt?
Oder lieber den Ungerüsteten mit dem kleinen Streitkolben?

Wir gehen im folgenden auf einige grundlegende Stellschrauben ein, die das Ilaris-Regelwerk in dieser Hinsicht bietet.
Generell gilt dabei: Im besten Fall kann man aus der Beschreibung der SL erahnen, dass ein Gegner über bestimmte Fähigkeiten verfügt.
Ein dem Aussehen nach simples Skelett, das mich one-hittet, weil es mit 12x \emph{starke Offensive} ausgestattet wurde, macht keinen Spaß.
\platz


\unterabsatz{Offensive}
Ein offensivstarker Gegner in Ilaris zeichnet sich üblicherweise durch hohe AT (Attacke) oder hohen Schaden aus.
Du kannst deinen Spielern also Abwechslung bieten, indem du zwischen diesen Stellschrauben variierst. 
Bringe mal den langsamen Troll mit der riesigen Axt (hoher Schaden, moderate AT) und mal den geschickten Klingentänzer (moderater Schaden, hohe AT).
Darüber hinaus können Spezialmanöver deine SCs ins Schwitzen bringen. Ein erfahrener Kämpfer greift vielleicht gezielt Schwachstellen an (Rüstungsbrecher gegen Platte, Umreißen gegen niedrige Gewandheit, Entwaffnen gegen niedrige Körperkraft).
Auch \textit{Todesstoß} ist ein Manöver, das man auf dem Schirm haben sollte. Mehr dazu im Abschnitt Defensive. 

Auch übernatürliches Wirken kann starke Offensivfähigkeiten beinhalten.
Dabei ist zu sagen, dass \textit{Crowd Control} in Ilaris extrem stark und für Spieler schnell frustrierend ist.
Den ganzen Kampf aussetzen, weil man eine MR-Probe nicht schafft, macht eher wenig Spaß.
Also \biglittlecap{Höllenpein}, \biglittlecap{Paralysis} und dergleichen mit Bedacht einsetzen!

\spaltenende
\begin{center}
	\zeichnung[0.68\textwidth]{massenkampf.jpg}
\end{center}
\spaltenanfang

\unterabsatz{Defensive}
Analog zu AT und Schaden gibt es in der Defensive die Stellschrauben VT (\textit{Verteidigung}) und \textit{Wundschwelle}.
Auch hier macht Variation das Spiel interessanter.

Die \textit{Wundschwelle} hat dabei noch detailliertere Aspekte. Erstens wäre der \textit{Rüstungsschutz} zu nennen, den man mit \textit{rüstungsbrechend}en Waffen kontern kann. Und zweitens die Kreatureneigenschaft \textit{Koloss} (\ilaris{96}). Man kann die WS halbieren und dafür eine Stufe \textit{Koloss} geben. Die Regeln lesen sich so, als würde man das nur machen, damit die WS nicht unmöglich hoch wird. Tatsächlich hat es aber einen weiteren ganz entscheidenden Effekt: Es halbiert den Schaden aus \textit{Todesstoß}. Wenn du also findest, dass \textit{Todesstoß} zu sehr dominiert, bringe öfter mal \textit{Koloss}e mit niedrigerer \textit{Wundschwelle}. Dies ist zum Beispiel passend bei Schwärmen, fettleibigen aber labbrigen Zombies oder amorphen Schleimungeheuern. 
\neuespalte

\mbox{}

\platz

Für die Verteidigung gegen Fernkampf braucht es eine Schild-VT oder eine Akrobatik Probe (28). Leider fehlt ein Wert dafür bei fast allen Kreaturen im Bestiarium. Wenn unklar, würde ich die schlechteste Nahkampf-VT verwenden. Gibt es in deiner Gruppe starke Fernkämpfer, kannst du ihnen hin- und wieder Gegner mit hoher Schild-VT entgegenstellen.

\smallskip

Ein weiterer sehr relevanter Defensivwert ist die \textit{Magieresistenz}.
Wenn dein Magier sich nutzlos fühlt, setze ihm mal einen Leibwächter vor, der kaum zu treffen ist, den er aber spielend paralysiert.
Wenn andererseits der Magier zu oft den Kampf entscheidet, bringe mal einen starken Gegner mit MU 10 und den zugehörigen Antimagievorteilen (\textit{Willensstark I+II}, \textit{Aurapanzer}, \textit{Unbeugsamkeit}), vielleicht auch \textit{Magieabweisend}.



\unterabsatz{Reichweite und Geschwindigkeit}
Nicht zu vergessen ist der Aspekt der Distanz. Dazu gehören vor allem \textit{Reichweite} und \textit{Geschwindigkeit}.

 Fernkämpfer aller Art sind in Ilaris bis in sehr hohe Stufen brandgefährlich. Das liegt daran, dass die \textit{Verteidigung} eine fixe Schwierigkeit 28 ist.
 Zum Treffen wird allerdings nur eine 12 benötigt (wenn auch oft mit Erschwernis). Ein herausragender Langbogenschütze mit \textit{Meisterschuss} verursacht zum Beispiel auf 256 Schritt 2 Autowunden.
 Dafür bekommt er zwar -16 auf den Angriff (2x \textit{Reichweite erhöhen} -8, \textit{Meisterschuss} -8), die Verteidigung geht aber trotzdem gegen 28.
 Vielleicht solltest du es also mit allzu guten (bzw. gut positionierten) Fernkämpfern nicht übertreiben.



\textit{Geschwindigkeit} wird in meiner Erfahrung von DSA-Spielern vor allem offensiv eingesetzt:
Ich renne auf den Gegner zu, am besten mit Sturmangriff!
Wenn man die Regeln ernst nimmt ist sie allerdings vor allem die beste Defensivoption.
Denn solange der Gegner einen nicht in Reichweite kriegt, kann er einem nicht weh tun.
Hat ein Gegner sowohl höhere Geschwindigkeit als auch höhere Reichweite, ist er schlicht unbesiegbar solange er kitet (also wegrennt und schießt). Ob man dies als Gegnertaktik einsetzen möchte, sei dahin gestellt, auch da übliche Battlemaps dafür viel zu klein sind. Gelegentlich kann es aber eine interessante (oder frustrierende) Herausforderung sein. Beachte hierbei auf jeden Fall die Regel zu \textit{Passierschläge}n.


\unterabsatz{Kampfdauer}
Typischerweise dauert ein Kampf in Ilaris 1--6 Runden, zu je 4 ingame-Sekunden.
Beachte dies bei deinen Überlegungen zu Zwischenzielen: Sind diese in der kurzen Zeit überhaupt erreichbar?
Wenn nicht, eigenen sie sich vermutlich nicht, um während dem laufenden Kampf verfolgt zu werden.
Auch Zauber mit längerer Vorbereitung sind meist eine schlechte Idee und daher eine Quelle für Frust auf Spielerseite.
Für Gegner bieten sie sich dagegen an um ein im Zeitrahmen erreichbare Zwischenziel zu setzen:
Störe die Konzenration des Magiers!


Du kannst Kämpfe natürlich in die Länge ziehen, zum Beispiel durch stetig eintreffende Verstärkungstruppen, besonders zähe Gegner oder eine riesige Battlemap auf der erst einmal 100 Felder Sumpf überwunden werden müssen.
Hierbei solltest du allerdings im Auge behalten, dass eine Runde durchaus 10 Minuten am Tisch in Anspruch nehmen kann.
Der Kampf sollte sowohl taktisch als auch erzählerisch genug bieten um seine outgame-Dauer zu rechtfertigen!
%
%\zeichnung[0.8\columnwidth]{Drache.jpg}

\unterabsatz{Anzahl der Gegner}
Es kann schwierig sein, einzelne Gegner zu designen, die nicht pro Runde einen SC töten, aber trotzdem eine Chance haben, den Kampf zu gewinnen. Das erklärt sich etwa so: Ilaris-Kämpfe sind darauf ausgelegt ca. 2--4 Runden zu dauern. Wenn dein Endgegner gegen 5 SCs gewinnen können soll, muss er im Schnitt also 1,25 pro Runde ausschalten.

 Darüber hinaus ist ein einzelner Gegner etwas langweilig, weil die Entscheidung \enquote{auf wen haue ich drauf?} nicht vorhanden ist.
 Daher ist es meist eine gute Idee, Endgegnern einige Minion zur Seite zu stellen.
 Ausnahmen bestätigen wie immer die Regel. 

Die Frage danach, wie viele Gegner es sein dürfen, stellt sich aber auch generell.
Es kann eine gute Idee sein, schon in der Anzahl für Abwechslung zu sorgen.
Vielleicht darf es mal ein starker Drache alleine sein, mal eine konkurrierende Abenteurerbande, und mal eine Übermacht von schwachen Untoten.
Vielleicht auch mal ein starker Drache mit einer Armee von schwachen Untoten gleichzeitig.
Denn allein durch die Gegnerzahl erzeugst du unterschiedliche regeltechnische und taktische Herausforderungen.
Gegen die Übermacht von Untoten kann zum Beispiel der Charakter mit \textit{Kraftvoller Kampf III} den \textit{Befreiungsschlag} voll auskosten, gegen einen einzelnen Drachen dagegen ist dieser nutzlos.
Gegen eine Überzahl will man eine Engstelle halten, eine Unterzahl will man dagegen umzingeln.


Beachte die Reaktionmali für mehrere Verteidigungen. Dadurch können auch schwächere Gegner in Überzahl eine ernste Gefahr darstellen. Greifen zum Beispiel 3 Gegner einen SC an, hat der Dritte effektiv +8 auf seine AT.
Dies gilt natürlich auch anders herum.
Wer die genannten Reaktionsmali nicht ignorieren kann, wird von einer kampfstarken Heldengruppe in einer Runde weggeputzt, auch mit VT 24.

\unterabsatz{Endgegner}
Möchtest du deinen Spielern trotz obigen Bedenken einen einzelnen Gegner vorsetzen, muss dieser vor allem defensiv stark genug sein, um einige Runden durchzuhalten. Am besten gibst du ihm mindestens eine Stufe \textit{Koloss}, da er sonst zu schnell von Todesstößen zerlegt wird.
Alternativ hilft auch eine hohe Verteidigung in Kombination mit größerer Größenkategorie (durch diese erhält er keine Reaktionsmali).
Offensiv wird es wie oben beschrieben schwierig.
Da dein Gegner im Vergleich zu der Gruppe so selten dran kommt, muss er jedes Mal wirklich was reißen. %Du kannst dies etwas auflockern, indem du als Hausregel die Aktionen des Gegners gleichmäßig über die Kampfrunde verteilst. Ein Shruuf mit Zusätzliche Attacke IV könnte Beispielsweise einmal in seiner eigenen Iniphase angreifen, und dann jeweils am Ende jeder der Iniphasen der vier Spielercharaktere. So milderst du folgende Situation etwas ab. : \enquote{Ich laufe zu dem Dämon und haue ihn. Ok, [...] du machst ihm eine Wunde. Er haut zurück. Und er haut nochmal zurück. [...] Du hast jetzt dann -16 auf VT, oder? Er haut nochmal...} Stattdessen haben mit unserer Hausregel die anderen SCs noch die Möglichkeit einzuschreiten, bevor ihr Kamerad zerschreddert ist. 


Das Ilaris Regelwerk selbst bietet spezielle Regeln für Endgegner an (\ilaris{96}). Schwachstellen und eine Unterteilung in unterschiedliche Trefferzonen können die Entscheidung \enquote{wo haue ich drauf} wieder einführen.
Als Hausregel kann man die Idee noch etwas weiter führen und die Trefferzonen (regeltechnisch) als einzelne Gegner mit unterschiedlicher Initiative sehen.
So könntest du einen mächtigen Drachen in Kopf (Ini 12, Zauber, Biss oder Feueratem), Flügel (Ini 8, Flügelangriff zum Zurückstoßen oder Fliegen), Schwanz (Ini 6, Schwanzschlag) und Körper (Ini 2, Laufen mit Trampelangriff) aufteilen.
Dies verhindert, dass ein SC durch mehrere Attacken zerlegt wird, bevor auch nur ein einziger Mitspieler eine Chance hat etwas dagegen zu unternehmen, wie es vermutlich der Fall wäre, wenn der Drache 4 Aktionen auf einmal hätte.



Ein weiterer Kniff bei Endgegnern kann ein Kampf in Phasen sein.
Ein Thargunitoth-Paktierer könnte sich zum Beispiel als Untoter erheben, sobald er erschlagen wurde.
Dies garantiert einen gewissen Schutz gegen \textit{Burst}, also sehr viel Schaden auf einmal, der deinen Bösewicht eventuell in einer Runde ausschaltet.


\absatz{Interessante Umgebung}
Neben unterschiedlichen Verhalten und regeltechnischer Ausgestaltung der Gegner kann auch die Umgebung für Abwechslung in deinen Kämpfen sorgen.
Wenn deine Spieler kreativ sind, musst du vielleicht gar keine Details herausgeben.
Manchmal reicht es zu sagen, dass ihr in einem Alchemistenlabor kämpft.
Viele Spieler brauchen allerdings einen Stoß in die richtige Richtung. 

\unterabsatz{Terrain}
Allzu große Battlemaps sind unpraktisch in der Handhabung.
Auch sind große Wiesenflächen oder ähnliches nicht unbedingt besonders spannend.
Daher ist es eine Art Kunst geworden auf einer 20x20 bis 30x30 Felder große Karte ein möglichst abwechslungsreiches Schlachtfeld unterzubringen. Dabei kannst du Hindernisse (Terrain, Wände, ...) verschiedener Art einsetzen um dem Kampf taktische Tiefe zu geben. 

Um ein Hafenbecken herum ist die Lauf-Distanz zum Beispiel eine andere, als die Bogenschuss-Distanz.
Soll ein feindlicher Magier am Bergpass nicht trivial zu erreichen sein, kannst du ihn weit oben am Hang positionieren, oder aber einfach auf einen Felsen setzen, den man erklettern muss.
Im besten Fall überlegst du dir vorab, wie und ob deine Hindernisse überwindbar sind und ob sie auch die Sicht- und Schusslinie versperren.
Unüberwindbar wäre zum Beispiel eine Höhlenwand. Ein Seidenvorhang dagegen blockiert nur die Sicht, ein Wasserbecken kann man durchschwimmen oder überfliegen, ein Geröllfeld halbiert die Bewegungsreichweite, eine Tür kann man einschlagen.

 Versuche mit solchen Elementen ein interessantes Schlachtfeld zu gestalten, das auf deine Battlemap/deine Skizze/deinen Tisch passt, und trotzdem genug Raum für taktische Bewegung liefert. 

\unterabsatz{Objekte}
Konkret genannte oder gemalte Objekte wie herumstehende Kisten, Krüge, Kronleuchter oder ähnliches können deine Spieler auf Ideen bringen.
 Vielleicht können sie als Deckung oder für untypische Kampfaktionen genutzt werden.
 Manche können sogar ein Zwischenziel im Kampf darstellen, wie größere Mengen Brandöl, eine Alarmglocke oder der Opferaltar mit angeketteter Jungfrau.
 Versuche daher zu Kampfbeginn mit wenigen Worten möglichst viele Details in der Vorstellung deiner Spieler zu erwecken.
 
 Im besten Fall sind die Objekte in einer vorbereiteten Battlemap eingezeichnet, oder du hast sogar ein 3-dimensionales Modell gebaut. Dies ist jedoch oft zu viel Aufwand, vor allem wenn nicht sicher ist, ob und wo genau der Kampf stattfinden wird. Aber auch auf einer Skizze kannst du Dinge deutlich machen und gegebenenfalls durch 3D-Objekte wie Jengasteine hervorheben.

\unterabsatz{Vorgefertigtes Material}
Das Internet ist voll von vorgefertigten Battlemaps. Manche von professionellen \textit{Content Creators}, andere von Hobby P\&P-Spielern. Nicht alle werden deinen Geschmack treffen, aber es lohnt sich oft ein kurzes Googeln, ob es schon was passendes gibt.
Meist hat sich der Ersteller viel Arbeit gemacht, inklusive Gedanken zu interessantem Terrain und anregenden Objekten.
Arbeit, die du dir sparen kannst!

Für größere Schlachten eigenen sich auch die offiziellen Stadtpläne für aventurische Städte, zum Beispiel während einer Belagerung.
Wenn Initiative ausgerufen wird, kannst du dann für entsprechende Ausschnitte eine Skizze in passendem Maßstab anfertigen.

\platz

\zeichnung{zwerg.JPG}

\platz

\absatz{Wie viel darf es sein?}
Wir haben einige Gedanken entwickelt, was wir in einen Kampf werfen könnten, um diesen spannend und abwechslungsreich zu gestalten.
Das heißt aber nicht, dass es eine gute Idee ist, alles davon in einem einzigen Encounter unterzubringen.
Zum einen haben wir schon angesprochen, dass die Szene auch irgendwie in die Spielwelt passen sollte.
Eine winzige Blütenfee (hohe VT, niedrige WS) auftauchen zu lassen, nur um einen Kontrast zum Fleischgolem (niedrige VT, hohe WS) zu haben, fänden viele Rollenspieler wohl unpassend.
Aber fast wichtiger ist noch eine andere Einschränkung:
Du und deine Spieler seid Menschen mit endlicher Hirnkapazität. Wenn du das gesamte Regelwerk in einem Kampf abfeuern willst, mit 13 verschiedenen Gegnertypen, etlichen Zwischenzielen, verschiedenstem Terrain und explodierenden Amphoren, wirst du deine Spieler überfordern -- und schon viel früher dich selbst.
Denn als Spielleiter musst du das ganze ja verwalten und managen. 


Wie viel man im Kopf behalten kann und will, ist sicher unterschiedlich.
Als Faustregel kann man vielleicht sagen, dass drei Gegnertypen, die sich wirklich unterscheiden und die unterschiedliche Rollen im Kampf einnehmen, eigentlich genug sind.
Vermeiden sollte man dagegen irrelevante Unterschiede, also zum Beispiel 5 Söldner, die alle irgendwie +1/-1 auf \textit{Attacke}, \textit{Verteidigung} oder \textit{Wundschwelle} haben.
 Solche marginalen Differenzen sind unsere Aufmerksamkeit nicht wert. Stattdessen lieber 2 Söldner in Kettenhemd mit Kriegshammer (Frontsau, einer zudem Anführer), 2 ungerüstet mit Bogen und Dolch (Artillerie) und einer mit zwei Schwertern in Leder (Assassine: Hält sich zu Beginn noch versteckt und macht dann Sturmangriff auf die hintere Reihe).
 
 \neueseite
 
  Auch bei interessanter Umgebung kann weniger mehr sein, wobei es dort oft kein Beinbruch ist, wenn man Elemente ignoriert/vergisst:
  Wenn keiner am Kronleuchter schwingen will, ist das eben so.
  Vermutlich heißt das, dass der Kampf auch ohne dieses Feature spannend war -- um so besser.
  Wenn du dagegen auf Gegnerseite einen Magier aufstellst, kannst du diesen schlecht einfach nichts tun zu lassen, weil dir gerade seine Zauber entfallen sind.

\bigskip

Speziell zu \textbf{Magiern} oder ähnlichen Gegnern:
Es kann sehr hilfreich sein, wenn du deren Zauberauswahl (für diesen Kampf) auf 1--3 beschränkst und die Regeln bereit hältst, zum Beispiel als Manöverkarten.
Klar, wahrscheinlich kann der NSC mehr, aber du als Spielleiter möchtest vermutlich nicht neben allem anderen auch noch den optimalen Zauber aus einer längeren Liste wählen, zudem auch noch in passender Modifikation.
Stattdessen setzen wir: Der Borbaradianer macht den \biglittlecap{Höllenpein} mit 1\,x\,\textit{Vorbereitung verkürzen} gegen einzelne Ziele, \biglittlecap{Tlalucs Odem} mit 2\,x\,\textit{Vorbereitung verkürzen}  als Flächenschaden oder er beschwört einen Dämon, wenn er glaubt, die Aktion Konzentration für die Beherrschung aufbringen zu können.
Das ist bereits einiges zum Durchlesen bei der Vorbereitung und wird uns reichen, wenn wir nebenher noch seine zwei Handlanger und die Hauschimäre spielen müssen. 

\bigskip

Statt also alles in einen Encounter zu packen, greife dir einige Aspekte heraus, die dir passend scheinen. Vielleicht kannst du dich dabei an der dramatischen Frage orientieren.
Oder du gibst deinem Encounter einen Titel wie \enquote{Schusswechsel am Hafen}, der dich bei der Vorbereitung (oder der Improvisation) leitet.
Dass dabei dann nicht alles zum Tragen kommt, was du im Repertoire hast, ist umso besser.
So hast du Ideen für weitere Kämpfe, und jeder wird sich anders anfühlen!

\spaltenende

\vfill

\begin{flushright}
	\zeichnung{hazzim.jpg}
\end{flushright}

\neueseite

\begin{center}

\abschnitt{Das Ritual der Reinigung}
	
	\usekomafont{subsection}
    Ein Kampf-Encounter für das kostenlose P\&P-Regelwerk {\link{https://ilarisblog.wordpress.com/downloads/}{Ilaris}}
    \normalfont \normalsize
    
    \credit{Autor}{Arne Strehlow}
   
\end{center}
\spaltenanfang

Hier stellen wir einen Encounter vor, der einige der oben genannten Tipps umsetzt.
Wir haben den Kampf in zwei Phasen unterteilt, so dass die SCs zwischendurch Zeit haben, sich zu sammeln oder Vorbereitungen zu treffen.
Zum Beispiel könnten sie sich \textit{rüstungsbrechend}e Waffen beschaffen, oder die Stadtwache konsolidieren und effektiv einbinden.

\textbf{Phase 1} dient in diesem Sinne als eine Art Ankündigung, die die SCs auf die Art der Bedrohung aufmerksam machen und Spannung aufbauen soll.

\textbf{Phase 2} ist dann das epische Finale. 

\unterabsatz{Encounter-Schwierigkeit}
Dieser Encounter ist für eine gut ausgestattete Heldengruppe mit Kampffertigkeiten um die 16 sowie \textit{Hammerschlag}/\textit{Todesstoß} im Repertoire gedacht.
Oder alternativ für eine kreative, die in Phase 2 vor allem die Stadtwache, die Stadtbewohner oder übernatürliches Wirken effektiv einsetzt.
Ohne einen Fernkämpfer oder ein übernatürliches Pendant ist mit Scheitern des Rituals zu rechnen.


\absatz{Das Ritual}
Um das Hafenbecken eines kleinen Städtchens von charyptider Verseuchung zu säubern, veranstaltet der Efferdtempel einen Gottesdienst, bei dem ein reinigendes Ritual durchgeführt werden soll.
Dutzende Gläubige haben sich am Pier versammelt um dem Spektakel beizuwohnen und mit ihren Gebeten zu unterstützen.

Doch was ist das? monströse Gestalten erheben sich aus dem Wasser. Hätte man damit nicht rechnen müssen? Die Geweihten sind mit dem Ritual beschäftigt, dass sie nicht ohne weiteres abbrechen können, da dafür bereits ein unersetzliches Opfer erbracht wurde (vielleicht ein heiliger Gegenstand, den die SCs zuvor beschafft haben).
Es braucht also tapfere Recken, um sie zu schützen.

\unterabsatz{Dramatische Frage}
Werden die Hummerier es schaffen, das Ritual zu stören? Dies tun sie, falls die Efferdgeweihten ihre Konzentration aufgeben müssen.

Falls es sich anbietet, machen die Hummerier nebenher auch gerne Beute in Form von Fleisch, Waffen oder Wertgegenständen. Gegen entschiedenem Widerstand braucht es allerdings den \textbf{Hummerier-Priester}, der als Einziger ein persönliches Interesse an der Mission hat.

\unterabsatz{Battlemap}
Dieser Encounter ist relativ generisch und soll flexibel einsetzbar sein. Am besten nutzt du den offiziellen Stadtplan der Stadt, in der du ihn verwendest, als Übersicht.
Für relevante Ausschnitte wie den Anleger kannst du eine Skizze in detaillierterem Maßstab anfertigen, oder einen herangezoomten Ausschnitt verwenden.
Vielleicht ergeben sich aus dieser Karte weitere taktische Ideen. Von einem benachbarten Anleger könnte ein Fernkämpfer zum Beispiel einen besseren Winkel auf kletternde Humemrier haben, muss sich dafür aber für eine Seite entscheiden. 
%\newline
%Schön wäre es natürlich, mit so einer Spielhilfe auch eine konkrete Beispielkarte zu präsentieren. Leider fehlen uns dafür die Künstler.

\unterabsatz{Regeltechnisches}
Ein zentrales Regelthema dieses Encounters ist hoher Rüstungsschutz, dem die SCs mit dem Manöver \textit{Rüstungsbrecher} (benötigt entsprechende Waffe) oder \textit{Todesstoß} beikommen können.
Beschreibe am besten gleich bei der ersten Sichtung den dicken, warzigen Panzer der Hummerier. Eine Probe auf \textit{Sagen \& Legenden} (16) kann offenbaren, dass die Biester unter dem Panzer relativ weich sind, etwa wie ein Mensch in Plattenrüstung. 

Mache dich ansonsten mit einigen Kampf-Vorteilen und übernatürlichen Talenten der Hummerier vertraut. Die wichtigsten könnten zum Beispiel sein: \textit{Offensiver Kampfstil} (\textbf{Soldat}), \textit{Schildwall} und \textit{Gegenhalten} (\textbf{Leibwächter}), \textit{Befreiungsschlag} und \textit{Ertränken} (\textbf{Priester}), \biglittlecap{Aquasphaero} (\textbf{Elementarist}). 

\unterabsatz{Phase 1}
Angeführt von einem \textbf{Leibwächter} versuchen drei \textbf{Soldaten} den Pier zu erklimmen, beziehungsweise ihre Kameraden dabei zu decken. Von oben hat man immerhin eine erhöhte Position und kann sie vielleicht daran hindern.
Dies versucht zumindest der örtliche Rondrageweihte mit seinem Knappen, während der adelige Stadtherr nur ruft: \enquote{Seid ihr wahnsinnig? Zieht euch vom Anleger zurück!}, und so die Stadtwache vorerst davon abhält, einzugreifen.
Die Efferdgeweihten jedoch scheinen willens, ihr Leben zu geben.
Oder sind sie so tief in ihre Gesänge versunken, dass sie die Gefahr nicht bemerken?
Sollten die Hummerier auf effektiven Widerstand stoßen, ziehen sie sich vorerst zurück. 
\smallskip 
\newline 
\minisec{Regeln}
Zum Erklimmen des Piers ist eine Aktion \textit{Konzentration} erforderlich. Beachte die Regeln zur Konzentration, insbesondere die Möglichkeit diese zu unterbrechen (\ilaris{37}).
Gelingt dies bevor der Kletterer das nächste mal am Zug ist, fällt er zurück ins Wasser. 

Vom Pier aus hat man eine \textit{überlegene} Position gegen Wesen im Wasser  (+4). Um von oben ins Wasser, oder umgekehrt, angreifen zu können, benötigt man allerdings Reichweite 2. (Kletterer können von oben und unten auch mit kürzeren Waffen erreicht werden.)

Wer an Rand des Piers stürzt, würfelt einen W6. Bei 1-2 endet der Sturz im Wasser. (Achtung, sehr tödlich. Wenn SC-Tode nicht gewünscht sind, vielleicht besser weglassen.)

\minisec{Gegner} (alles Hummerier) \newline
Stoßtrupp: 1 Leibwächter (Anführer, Beschützer, hat Befehl das Ritual zu stören, möchte aber gerne alle seine Jungs wieder sicher nach Hause bringen),

3 Soldaten (Frontsau)

\unterabsatz{Phase 2}
Sollten die ersten Hummerier keinen Erfolg haben, ziehen sie sich zurück und warten auf Verstärkung. Immer wieder sieht man ein Wesen in 100 Schritt Entfernung  kurz auftauchen und das Ritual beobachten. Die Flut scheint unnatürlich schnell zu kommen (der Hummerier-Priester setzt \enquote{Gebieter der Gezeiten} ein).

Nach etwa einer halben Stunde gehen zwei Hummerier-Ablenkungstrupps an einem einfacher zugänglichen Ufer an Land und greifen wahllos die Stadtbevölkerung an. Sie sollen Verteidiger vom Pier abziehen. Kurz darauf versucht ein Angriffstrupp erneut, den Pier zu erklimmen. 

Währenddessen heben sich in etwa 60 Schritt Entfernung Gestalten aus dem Wasser. Ein Hummerier-Elementarist formt dort einen \biglittlecap{Aquasphaero} (mit Zeit lassen, doppelter Wirkungsdauer und 3\,x\,\textit{Mächtiger Magie}, benötigt eine gewürfelte 5). Anschließend fliegt der Wasserball 4 INI-Phasen lang auf die Efferdgeweihten zu.
Der Hummerier muss sich derweil auf seinen Zauber \textit{konzentrieren}.
Sollte der Ball eintreffen werden die Geweihten vermutlich die Konzentration verlieren und das Ritual scheitert.

Der Elementarist schickt weitere \biglittlecap{Aquasphaeri}, solange er nicht davon abgehalten wird.

\smallskip

Wir empfehlen die Ablenkungstrupps nur dann detailliert nach Kampfregeln auszuspielen, wenn die SCs direkt mit diesen interagieren.
Sonst kannst du deren Erfolg nach den Regeln des \textit{Massenkampf}s bestimmen (\ilaris{60}), oder einfach frei nach SL-Entscheid erzählen.


\minisec{Regeln}
Der Pier gibt wegen dem höheren Wasserstand nur noch \textit{vorteilhafte} Position (+2) und es benötigt nur noch eine Aktion \textit{Bewegung} diesen zu erklimmen.

Der Elementarist benötigt \textit{Konzentration}, um seinen \biglittlecap{Aquasphaero} zu lenken (siehe \ilaris{139}).
Er kann also auch während der Wasserball fliegt noch gestört werden.
Beim zweiten Mal verwendet er dann allerdings eventuell die Modifikation \textit{vorgegebene Bewegung}.

\minisec{Gegner} (alles Hummerier)
\begin{itemize}
\item 2\,x\,Ablenkungstrupp: Ein Leibwächter (Anführer, Beschützer, hat Befehl durch Blutbad für Ablenkung zu sorgen, möchte aber gerne alle seine Jungs wieder sicher nach Hause bringen), 3 Soldaten (Frontsau)
\item 1\,x\,Angriffstrupp: 1 Priester (Anführer, Frontsau), 4 Soldaten (Frontsau)
\item 1x Magietrupp: 1 Elementarist (Artillerie, Zeitdrücker), 1~Leibwächter (Beschützer)
\end{itemize}

In diese Anzahl Hummerier sind jene aus Phase 1 eingerechnet. Sie sind dementsprechend eventuell verwundet oder nehmen sogar gar nicht mehr am Kampf teil. (Ab 4 Wunden schicken auch die Hummerier ihre Kameraden nicht zurück aufs Schlachtfeld.) Der Priester teilt die Trupps entsprechend anders ein.

Sollte der Priester fallen, ist der Rest der Hummerier nur noch mäßig motiviert und wird keine größeren Risiken für ihre finstere Göttin mehr eingehen.

\unterabsatz{Zwischenziele (Beispiele)}
\begin{itemize}
\item Die Stadtwache zum effektiven Eingreifen bewegen.
\item Die Stadtbewohner schützen.
\item Ein junger Mann wird von sich zurückziehenden Hummeriern ins Wasser gezogen und will gerettet werden (oder der Rondra-Knappe stürzt durch ein \textit{Umreißen}-Manöver vom Pier).
\item Den Elementaristen stoppen.
\end{itemize}

\unterabsatz{Anpassungen}
Leichter kannst du das Geschehen machen, indem du den Helden mehr Unterstützung von Seiten der Stadt zugestehst, oder weniger Hummerier angreifen lässt.
Auch könntest du den Elementaristen ohne Leibwächter an der Seite auftreten lassen. 

Sollte es sich um eine größere Stadt handeln, sind sowohl mehr Kampfkraft der Stadt als auch mehr Angreifer eventuell angemessen.
Wie du größere Schlachten abhandelst, ist allerdings ein Thema für sich.
Du kannst zum Beispiel nur die Aktionen der Helden detailliert ausspielen und für den Rest der Schlacht alle paar Kampfrunden eine Probe nach den \textit{Massenkampf}regeln (\ilaris{60}) ablegen. 


Um es schwerer zu machen könnte der Elementarist einen Dschinn des Unwassers gebunden haben, der für ihn in den Kampf eingreift. 

Wenn du für mehr Abwechslung unter den Gegnern sorgen möchtest, kannst du natürlich auch andere Wesen einbauen. Vielleicht zum Kontrast welche mit niedrigem Rüstungsschutz? Unwasserelementare oder charyptide Dämonen könnten passen. Weitere Möglichkeiten wären aufgeweichte Wasserleichen, ungerüstete Krakonier oder ein Krakenmolch.

Auch könnte es Verräter in der Stadt geben, zum Beispiel einen charyptiden Kult. Frage dich aber, ob du dich selbst nicht mit so vielen unterschiedlichen Gegnertypen überforderst.

Schließlich musst du alle im Kampf managen!

\newpage

\absatz{Hummerier}
Hummerier sind eine Spezies der Vielbeinigen. Zumindest fassen wir sie hier so auf und klassifizieren sie als Mythenwesen. In der Spielwelt treten sie jedoch auch als Daimonid auf. Eine ausführlichere Beschreibung findest du zum Beispiel in der Spielhilfe \textit{Efferds Wogen}.
%\smallskip

\link{https://de.wiki-aventurica.de/wiki/Hummerier_(Spezies)}{Hummerier} sehen aus wie aufrechtgehende Hummer, haben einen dicken roten Panzer, Stielaugen und Fühler, vier Arme (davon zwei mit Scheren), einen gepanzerten Schwanz und blaues, tintenähnliches Blut.
Zu einer Lautsprache sind sie nicht fähig, nichtsdestoweniger sind sie in der Lage, sich untereinander zu verständigen. An Land bewegen sie sich langsam und ungelenk.
%\smallskip

Wir stellen hier eine Auswahl an Hummeriern vor, die dazu gedacht sind, eine fordernde Kampfbegegnung darzustellen. Generell ist zu beachten, dass ein \textit{Kampf im Wasser} den Schwierigkeitsgrad enorm erhöht. Außerdem haben diese Hummerier einen sehr hohen Rüstungsschutz. Als Warnung könntest du Gerüchte über ein Monster mit undruchdringlichem Panzer streuen. Auch könntest du die Gruppe mit einem kleinerern Kampf gegen Krabben in hüfttiefem Wasser auf die entsprechenden Gefahren und Regelmachnismen frühzeitig aufmerksam machen.
%\smallskip

Ein \textbf{Hummerier-Soldat} ist gewissermaßen die Standardausführung der Hummerier. Für eine unerfahrene Heldengruppe ist er trotzdem bereits alleine und an Land ein ernstzunehmender Gegner. Im Wasser ist er selbst für erfahrene Kämpfer eine tödliche Gefahr. 
%\smallskip

Der \textbf{Hummerier-Priester} führt einen kleinen Trupp Hummerier auf schwierigen Missionen an. Er kann seine Untergebenen auch dazu bringen, ein Stück ins Landesinnere vorzustoßen. Sollte er fallen, verlieren die restlichen Hummerier ihr Missionsziel aus den Augen und ziehen sich bald ins nächste Gewässer zurück. 
%\smallskip

Der \textbf{Hummerier-Elementarist} befindet sich gerne in schwer zu erreichender Position -- zum Beispiel schwimmend im Meer bei einem Angriff auf einen Hafen. Von dort webt er \biglittlecap{Aquasphaero}s und lenkt diese (braucht Konzentration), hebt die Scheren zur Anrufung eines Elementars, oder ähnliches. Dies soll die SCs vor die Entscheidung stellen, ob und wie sie ihn beim Zaubern stören. Dabei kann man den SCs mit \textit{Vorbereitung verkürzen}/\textit{Zeit lassen} das Unterbrechen schwerer/einfacher machen. 
%\smallskip

Die \textbf{Hummerier-Leibgarde} schließlich stellt solide Defensivkämpfer dar. Sie haben eine Schere mit Tang, Treibholz und Korallen so umwickelt, dass sie diese als \textit{Schild} nutzen können (aber nicht mehr für den \textit{Doppelangriff}). Um Elementarist und/oder Priester zu schützen, verwenden sie gerne \textit{Defensiver Kampfstil} und \textit{Schildwall}.

\neuespalte

% Aber Vorsicht! Wenn du das volle Arsenal auffährst, also einige Soldaten angeführt von Priester mit Leibgarde und aus der Ferne unterstützt von Elementarist mit Leibgarde, dann nimmst du schnell deine Heldengruppe auseinander. Achte darauf, dass die Kampfschwierigkeit der Situation angemessen ist und das für die SCs ersichtlich ist, auf was sie sich einlassen. An Land können sie zumindest ganz gut weglaufen.

\kreaturhummeriersoldat
\kreaturhummerierleibgarde
\kreaturhummerierelementarist
\kreaturhummerierpriesterkrustentiere
\spaltenende