\credit{Autor}{Alrik Normalpaktierer}
\spaltenanfang
Im \link{https://dsaforum.de/app.php/dlext/details?df_id=416}{ersten Band der Chroniken von Ilaris}
hatten wir neben neuen Spielhilfen (wie der ersten Version der Manöverkarten, einigen Tipps zum Abenteuervorbereiten, \dots) auch einen Überblick über kostenfreien Spielhilfen zusammengestellt, die den Einstieg in Ilaris vereinfachen.
Diese Übersicht ist immer noch hilfreich, aber vom Stand Früh\-som\-mer 2022.
Da die Entwicklung vieler Projekte weitergelaufen ist, nicht mehr in allen Punkten aktuell.
Inzwischen bietet \link{https://ilaris-online.de/app/inhalte/}{Ilaris Online} eine vollständige Liste, filter- und sortierbar nach Kategorien. Damit ergänzt die Website das Angebot des gesamten Ilaris-Regelwerks als Datenbank.

Im Folgenden wollen wir einige ausgewählte Spielhilfen detaillierter vorstellen.

\subsection*{Generierungsprogramm Sephrasto}
\link{https://github.com/Aeolitus/Sephrasto/releases}{Sephrasto} hat in den letzten Jahren erneut eine deutliche Weiterentwicklung durchlaufen und liegt inzwischen in der Version 5 vor.

\begin{itemize}
\item Vollumfassende regelkonforme Charaktererstellung und -steigerung mit zahlreichen Hilfestellungen
\item 	Assistent für die Erstellung neuer Charaktere
\item	Speichermöglichkeit und PDF-Export
\item	Automatisch erstellter PDF-Anhang mit allen für den Charakter relevanten Regeln
\item	Vier verschiedene Charakterbögen zur Auswahl, auch eigene sind möglich
\item	Umfassende Unterstützung für Hausregeln durch einen ins Programm integrierten Datenbankeditor. Hausregeln können so auch einfach gespeichert und geteilt werden.
\item	Sephrasto unterstützt Plugins. Sie lassen sich aus dem Programm heraus installieren und (de-)aktivieren.
Zahlreiche von der Community erstellte Plugins -- z.\,B. das Manöverkarten-Tool (s.\,u.) und ein Exporter für Foundry-VTT -- stehen zur Verfügung.
\item	Grafische Darstellung über Schriftarten und -größen und Themes leicht anpassbar
\item	In-App-Hilfe
\end{itemize}

\zeichnung[0.8\columnwidth]{Curthan2.jpg}



\subsection*{Manöverkarten}
Eines der Plugins erlaubt, einen Satz Manöverkarten passend zur jeweiligen Figur zu erstellen.
Dazu gehören die verfügbaren Manöver im Kampf und praktische Regelübersichten.
Auch der Regeltext aller Zauber und Liturgien liegen dabei auf jeweils einer Karte mit sprechenden Symbolen vor.
So erhältst du bereits mit dem Erstellen deines Charakterbogens deine Optionen im Spiel druckfertig in übersichtlicher Form.

\subsection*{Kreaturendatenbank}
Auch die \link{https://ilaris-tools.github.io/IlarisDB/db/kreaturen/}{Kreaturen\-datenbank} wurde erheblich weiterentwickelt.
Sie umfasst neben allen Kreaturen des Regelwerks inzwischen auch viele Gegner und NSCs, die andere Spielleitungen in ihrer Vorbereitung erstellt haben.
Spielwerte stehen so als Konvertierungen für Abenteuer anderer Regeleditionen zur Verfügung.

Angemeldete Nutzende können selbst Kreaturen eintragen und sie für FoundryVTT oder als LaTeX- oder als Brauerei-Code (s.\,u.) im Format einer Manöverkarte exportieren.

%In der Ausarbeitung eigener Szenarien oder der Vorbereitung des Spielabends ist die stets überarbeitete Spielleitung für folgende Hilfestellungen dankbar:
%
%\tabelle{p{1.9cm}|p{1cm} X X X}{
%	
%	\tkopf{Spielhilfe} & \tkopf{von} & \tkopf{Was ist das?} &\tkopf{Wobei hilft es?}  \\
%	\hline
%	
%	\link{https://ilaris-tools.github.io/IlarisDB/db/kreaturen/}{Kreaturen\-datenbank}&Lukas Ruhe&Eine Datenbank mit den Spielwerten aus dem Bestiarium und vielen Hausregeln.&Schnell Spielwerte nachzuschlagen und zu kopieren.\\
%	%		\item CharakterToText
%	
%	\hline
%}

\subsection*{Layout: Die Brauerei}


%\tabelle{p{3cm}|p{2.5cm} X }{
%	\tkopf{Spielhilfe} & \tkopf{von} & \tkopf{Was ist das?} \\
%	\hline
%	
%\link{https://github.com/Ilaris-Tools/IlarisTex}{IlarisTex}&Lukas Ruhe&\LaTeX-Klasse, die zahlreiche Gestaltungselemente zur Verfügung stellt.\\
%
%\link{https://www.dropbox.com/sh/xs1w5r4rq3m0a99/AAB2MGJvEr0O8uxG4_9USVpQa?dl=0}{Ilaris-Artwork}&Bernhard Eisner&Symbole und Illustrationen aus dem Ilaris-Regelbuch.\\
%
%\link{https://webzine.nandurion.de/2013/10/15/dsa-schriftartenpaket-stand-022011/}{Schriftarten-Paket}&Salaza&Digitale Fonts für verschiedene aventurische Schriften.\\
%\hline
%}
\spaltenende