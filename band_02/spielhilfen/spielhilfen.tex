\credit{Autor}{Alrik Normalpaktierer}
\spaltenanfang
\subsection*{Ilaris Online}
Im \link{https://dsaforum.de/app.php/dlext/details?df_id=416}{ersten Band der Chroniken von Ilaris}
hatten wir neben neuen Spielhilfen (wie der ersten Version der Manöverkarten, einigen Tipps zum Abenteuervorbereiten, \dots) auch einen Überblick über kostenfreie Spielhilfen zusammengestellt, die den Einstieg in Ilaris vereinfachen.
Diese Übersicht ist immer noch hilfreich, aber vom Stand Früh\-som\-mer 2022.
Da die Entwicklung vieler Projekte weiterlief, ist sie nicht mehr aktuell.

Inzwischen bietet euch \link{https://ilaris-online.de/app/inhalte/}{Ilaris Online} eine vollständige Liste, filter- und sortierbar nach Kategorien.
Auf Ilaris Online werden neben dem gesamten Ilaris-Regelwerk auch noch viele weitere nützliche Seiten und Tools angeboten.

Im Folgenden werden wir einige ausgewählte Ilaris-Projekte detaillierter vorstellen.

\subsection*{Kreaturendatenbank}
Die \link{https://ilaris-tools.github.io/IlarisDB/db/kreaturen/}{Kreaturen\-datenbank} wurde von \textbf{Lukr} erheblich weiterentwickelt.
Sie umfasst neben allen Kreaturen des Regelwerks inzwischen auch viele Gegner und NSCs, die andere Spielleitungen in ihrer Vorbereitung erstellt haben.
Spielwerte stehen so als Konvertierungen für Abenteuer anderer Regeleditionen zur Verfügung.

Außerdem lassen sich Kreaturen auch zu Abenteuern zuordnen, sodass mit einem Klick alle Kreaturen eines Abenteuers aufgelistet werden können. So haben wir die Links auf den einzelnen Abenteuertiteln hier im Band umgesetzt.

Angemeldete Nutzende können selbst Kreaturen eintragen und sie für FoundryVTT oder als LaTeX- oder als Brauerei-Code (s.\,u.) im Format einer Manöverkarte exportieren.


\zeichnung[\columnwidth]{Curthan2_2.jpg}

\columnbreak

\subsection*{Layoutprogramm: Die Brauerei}
Mit der \link{https://brauerei.ilaris-online.de/}{Brauerei} hat \textbf{Lukr} eine einfach zu bedienende, aber funktionsreiche online Layout-Hilfe erstellt, mit der Dokumente und Handouts im Ilaris-Layout hergestellt und verteilt werden können. Keine Werbung, keine Anmeldung nötig. Einfach kreativ werden. 

Bewahre deine Links gut auf, diese brauchst du um deine Gebräue wiederzufinden und weiterzuarbeiten. Hier findest du eine \link{https://brauerei.ilaris-online.de/share/thZtbIr1lFHh}{Beispiel-Brauerei}.

Lukas stellt die Brauerei in \link{https://www.youtube.com/watch?v=HLdT4mEAL2s&list=PLVioVhDcd4OS7wWlxEOxO3RoBzSRPh5KN&index=1}{diesem Video} vor. Vielleicht sagt euch auch die \link{https://youtu.be/Jj5S6jwzf1Y?t=2419}{SchelmSchau 137} zu, in welcher der Schelm die Ilaris Brauerei näher betrachtet.


\subsection*{Charaktergenerator: Sephrasto}
\link{https://github.com/Aeolitus/Sephrasto/releases}{Sephrasto} ist in den letzten Jahren
-- vor allem von \textbf{Gatsu} --
erneut deutlich weiterentwickelt worden und liegt inzwischen in der Version 5 vor.

Zu den Features gehören:
\begin{itemize}
	\item vollumfassende regelkonforme Charaktererstellung und -steigerung mit zahlreichen Hilfestellungen
	\item 	Assistent für die Erstellung neuer Charaktere
	\item	Speichermöglichkeit und PDF-Export
	\item	Automatisch erstellter PDF-Anhang mit allen für den Charakter relevanten Regeln
	\item	Vier verschiedene Charakterbögen zur Auswahl, auch eigene Bögen sind möglich
	\item	Umfassende Unterstützung für Hausregeln durch einen ins Programm integrierten Datenbankeditor. Hausregeln können so auch einfach gespeichert und geteilt werden.
	\item	Grafische Darstellung über Schriftarten und -größen und Themes leicht anpassbar
	\item	In-App-Hilfe
	\item	Plugin-Manager. Plugins lassen sich aus dem Programm heraus installieren und (de-)aktivieren.
\end{itemize}

Zahlreiche von der Community erstellte Plugins -- z.\,B. das Manöverkarten-Tool (s.\,u.) und ein Exporter für \emph{Foundry Virtual Tabletop} -- stehen zur Verfügung.


\subsection*{Manöverkarten}
Ein von \textbf{Gatsu} erarbeitetes Sephrasto-Plugin erlaubt, einen Satz Manöverkarten passend zur jeweiligen Figur zu erstellen.

Dazu gehören die verfügbaren Manöver im Kampf und praktische Regelübersichten.
Auch der Regeltext aller Zauber und Liturgien liegt dabei auf jeweils einer Karte mit sprechenden Symbolen vor.

So erhältst du bereits mit dem Erstellen deines Charakterbogens deine Optionen im Spiel druckfertig in übersichtlicher Form.



%In der Ausarbeitung eigener Szenarien oder der Vorbereitung des Spielabends ist die stets überarbeitete Spielleitung für folgende Hilfestellungen dankbar:
%
%\tabelle{p{1.9cm}|p{1cm} X X X}{
%	
%	\tkopf{Spielhilfe} & \tkopf{von} & \tkopf{Was ist das?} &\tkopf{Wobei hilft es?}  \\
%	\hline
%	
%	\link{https://ilaris-tools.github.io/IlarisDB/db/kreaturen/}{Kreaturen\-datenbank}&Lukas Ruhe&Eine Datenbank mit den Spielwerten aus dem Bestiarium und vielen Hausregeln.&Schnell Spielwerte nachzuschlagen und zu kopieren.\\
%	%		\item CharakterToText
%	
%	\hline
%}



%\tabelle{p{3cm}|p{2.5cm} X }{
%	\tkopf{Spielhilfe} & \tkopf{von} & \tkopf{Was ist das?} \\
%	\hline
%	
%\link{https://github.com/Ilaris-Tools/IlarisTex}{IlarisTex}&Lukas Ruhe&\LaTeX-Klasse, die zahlreiche Gestaltungselemente zur Verfügung stellt.\\
%
%\link{https://www.dropbox.com/sh/xs1w5r4rq3m0a99/AAB2MGJvEr0O8uxG4_9USVpQa?dl=0}{Ilaris-Artwork}&Bernhard Eisner&Symbole und Illustrationen aus dem Ilaris-Regelbuch.\\
%
%\link{https://webzine.nandurion.de/2013/10/15/dsa-schriftartenpaket-stand-022011/}{Schriftarten-Paket}&Salaza&Digitale Fonts für verschiedene aventurische Schriften.\\
%\hline
%}
\spaltenende