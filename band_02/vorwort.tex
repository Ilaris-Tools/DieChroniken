Der zweite Ilaris-Abenteuerwettbewerb im Winter 2023/2024 war ein voller Erfolg, und das verdanken wir in erster Linie den herausragenden Einsendungen, die uns erreicht haben.
Unsere Jury, bestehend aus \textbf{Niklas Bräuer} (Almanach der Abenteuer), \textbf{Sidonia von Nadoret} und \textbf{Christian Bathen} (Mistress \& Knight), \textbf{Yvonne \enquote{Schattenkatze}} (Orkenspalter) sowie mir, \textbf{Xaver Stiensmeier}, für das Ilaris-Team, hatte beim Studieren große Freude.
Um so mehr freue ich mich jetzt, dass ihr den fertigen Band in den physischen oder digitalen Händen halten könnt. Jedes einzelne Abenteuer reizt auf seine eigene Weise und auch wenn wir final eine Reihenfolge festlegen mussten, weisen alle Beiträge \textit{„Hohe Qualität“} auf (Wortspiel beabsichtigt).

Deshalb, ohne große Worte zu den Platzierungen:

\subsection*{1. Platz: In den Hallen des Bergkönigs (Matthias Ott)}
Dieses Abenteuer führt erfahrene Helden tief ins Herz der Finsterkoppen, die Heimat von Bergkönig Garbalon Sohn des Gerambalosch, dem Hochkönig des Finsterkamms, dessen Zustand auch im Mittelpunkt der Handlung steht.

Unsere Helden müssen nicht nur ihre Kampffertigkeiten, sondern auch ihren Verstand in besonders praktischer Weise einsetzen, um das Königreich vor einer drohenden Gefahr zu bewahren.
\subsection*{2. Platz: Die Schwarze Hand im Wen\-gen\-holmer Land (Tilmann Kircher)}
In diesem „hotzenplotzerischen“ Abenteuer erwartet die Helden im Wengenholmer Land eine scheinbar einfache Aufgabe, die sich als schwerer als erwartet entpuppt, und deren Auflösung eine besondere Begegnung bereithält.
\subsection*{3. Platz: Asche im Wind\newline (Jasmin \& Benjamin Drogat)}
Dieses durchaus komplexe Sandbox-Abenteuer entführt die Spieler in das Svelttal, wo ihnen je nach Charakter ein anderer geschickter Einstieg ins tatsächliche Abenteuer gegeben wird: Sie müssen die Ursache für die unheimlichen Entität finden, die die Bewohner des Svelltlandes heimsucht. \textit{Asche im Wind} erzählt sich vor allem durch scheinbar zufällige Begegnungen und webt damit eine starke Atmosphäre. 

Eine reizvolle Aufgabe für erfahrene Spielleitungen.

\neuespalte

\subsection*{4. Platz: Klamm und Heimlich (Mathias Thanos)}
In diesem auf Neueinsteiger ausgerichteten Abenteuer zieht eine orkische Entführung nahe Greifenfurt euch sofort ins Geschehen. Wer, wenn nicht ihr, kann den Knappen Alfdan aus den orkischen Klauen (Hauern?) befreien? Doch gilt auch das Gebot der Vorsicht und damit des flinken und leisen Phex, da der Feind die Übermacht hat.

\medskip

In diesem Band präsentieren wir die Abenteuer geordnet nach Einstiegsfreundlichkeit und EP-Anforderung. So soll ein einfacher Einstieg und Dranbleiben gewährleistet werden. Falls ihr Abenteuer aus „Die Chroniken von Ilaris, Band I“ noch nicht gespielt habt, lassen sich diese gut nach \textit{Die Schwarze Hand im Wengenholmer Land} positionieren.
\bigskip

\subsection*{Spielhilfen}
Der Band wird durch einen Überblick über die neusten Spielhilfen abgerundet und bringt selbst noch die geniale Spielhilfe „Encounter-Design“ von Arne Strehlow mit. Encounter-Design erklärt, wie man Kämpfe spannender und vielschichtiger gestalten kann. Durch das Kampfszenario \textit{Das Ritual der Reinigung} wird die Theorie sogleich auch in die Praxis und damit an den Spieltisch überführt.% TODO Labellink setzen

Wir freuen uns über die gesteigerte Teilnehmerzahl im Vergleich zur letzten Durchführung.
Besonders möchte ich mich bei den Sponsoren Ulisses Spiele, René Dudziak, Matthias Ott und dem Ilaris-Team bedanken, die für großartige Preise gesorgt haben. Wir konnten also auch diesmal nicht alle Preise loswerden.
Vielleicht hilfst Du uns ja beim nächsten Mal dabei!

Ein besonderer Dank gilt auch den Künstlern Bernhard Eisner, Alnus (Anton Dobsak), Zillie Zeh und Matthias Thanos, die es uns ermöglicht haben, die Abenteuer visuell ansprechend zu gestalten.
Und natürlich gebührt ein riesiges Dankeschön dem Ilaris-Team, ohne das dieses Regelwerk nie das Licht der Welt erblickt hätte:
\textbf{Lukas \enquote{Curthan} Schafzahl, Lennart Sobirey, René Dudziak} und \textbf{Julius Natrup}.

Wo ich im Vorfeld viel organisatorische Arbeit hatte, kann ich mich nun, da ich diese Zeilen schreibe ziemlich zurücklehnen. Dank \textbf{Lukas Ruhe} und \textbf{Matthias Ott}, der bereits den ersten Wettbewerb organisiert und die \enquote{Chroniken von Ilaris -- Band I} herausgegeben hat, verläuft der Satz auch diesmal reibungslos. Gemeinsam – so sage ich mal ganz hochgestochen – haben wir es geschafft, den Wettbewerb auf ein neues Niveau zu heben.

Ich freue mich schon jetzt auf den nächsten Wettbewerb und bin gespannt, welche kreativen Abenteuer uns dann erwarten.

\neueseite